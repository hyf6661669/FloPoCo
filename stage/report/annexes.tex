	LTI filters in general are usually defined as sums of products.
	Several quantities are useful to understand their characteristics

	\subsection{Notations}
		The entire report will use several notations and conventions:
		\begin{itemize}
			\item in signal processing, t is the common notation for continue time and k the notation for discrete time. This report keeps this convention.
			\item $\langle\langle\mathcal{H}\rangle\rangle _{WCPG}$ is the worst case peak gain of the filter $\mathcal{H}$
			\item y is an output variable
			\item x is a state variable
			\item t is an intermediate variable
			\item a bold symbol (for example $\boldsymbol{h}$) denotes a matrix, that may be a vector
			\item a normal symbol (for example $\varepsilon_{t_1}(k)$) denotes a single value
		\end{itemize}

	\subsection{Reminders about signal processing}
	\begin{thdef}\label{sig} (Signal)
		Generally, a signal is a temporal variable, which takes a value from $\mathbb{R}$ at each time t.
		x(t) denotes the value of the signal x at the instant t.
		When dealing with discrete time events, the time will be represented by k.
		Then x(k) is said to be a sample.
		$\{x(k)\}_{k \geq 0}$ denotes all the values possible for the signal x.
		The rest of this report, adresses vectors of signals $\textbf{x}$, where $\textbf{x}(k) \in \mathbb{R}^{n}$
	\end{thdef}

	\begin{thdef}\label{l_1} ($\ell^1$ norm)
		The $\ell^1$ norm of a scalar signal x, denoted $\|x\|_{\ell^1}$, is the sum of absolute values of $x(k)$ at each instant k:
		\begin{equation} \label{l1}
				\|x\|_{\ell^1}=\sum_{k=0}^{+\infty}|x(k)|
		\end{equation}
		This norm exists only if $x$ is $\ell^1$-sommable, that is, if and only if the equation \ref{l1} converges.
	\end{thdef}
	
	\begin{thdef}\label{l_1} ($\ell^2$ norm)
		The $\ell^2$ norm of a scalar signal x, denoted $\|x\|_{\ell^2}$, is defined as follows:
		\begin{equation} \label{l2}
				\|x\|_{\ell^2}=\sqrt{\sum_{k=0}^{+\infty}x(k)^2}
		\end{equation}
		This norm exists only if $x$ is square-sommable, that is, if and only if the equation \ref{l2} converges.
	\end{thdef}

	\begin{thdef}\label{l_inf} ($\ell^\infty$ norm)
		The $\ell^\infty$ norm of a scalar signal x, denoted $\|x\|_{\ell^\infty}$, is the smallest upper bound among all values (absolute values) possible for the signal x, that is:
		\begin{equation} \label{l3}
				\|x\|_{\ell^\infty}=\sup_{k\in \mathbb{N}}|x(k)|
		\end{equation}
	\end{thdef}

		\begin{thdef}\label{Dirac} (Dirac)
			The Dirac function, denoted $\delta$ describes an impulsion of infinite amplitude at a discrete time k (supposed to be instaneous):
			\begin{equation} \label{ir}
					\delta(k)=
					\begin{cases}
						1 & when\  k=0 \\
						0 & else
					\end{cases}
			\end{equation}
				
		\end{thdef}
