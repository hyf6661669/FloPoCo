	LTI filters in general are usually defined as sums of products.
	Several quantities are useful to understand their characteristics

	\subsection{Notations}
		During the entire report we will use several notations and conventions:
		\begin{itemize}
			\item in signal processing, t is the common notation for continue time and k the notation for discrete time. This report keeps this convention.
			\item $\langle\langle\mathcal{H}\rangle\rangle _{WCPG}$ is the worst case peak gain of the filter $\mathcal{H}$
			\item y is an output variable
			\item x is a state variable
			\item t is an intermediate variable
			\item
			\item
		\end{itemize}
	\subsection{Remainders about signal processing}
	\begin{thdef}\label{sig} (Signal)
		Generally, a signal is a temporal variable, which takes a value from $\mathbb{R}$ at each time t.
		We denote x(t) the value of the signal x at the instant t.
		When dealing with discrete time events, the time will be represented by k.
		Then we talk about x(k), which is said to be a sample.
		$\{x(k)\}_{k \geq 0}$ denotes all the values possible for the signal x.
		In the rest of this report, we will talk about vectors of signals $\textbf{x}$, where $\textbf{x}(k) \in \mathbb{R}^{n}$
	\end{thdef}

	\begin{thdef}\label{l_1} ($\ell^1$ norm)
		The $\ell^1$ norm of a scalar signal x, denoted $\|x\|_{\ell^1}$, is the sum of absolute values of $x(k)$ at each instant k:
		\begin{equation} \label{l1}
				\|x\|_{\ell^1}=\sum_{k=0}^{+\infty}|x(k)|
		\end{equation}
		This norm exists only if $x$ is $\ell^1$-sommable, that is, if and only if the equation \ref{l1} converges.
	\end{thdef}
	
	\begin{thdef}\label{l_1} ($\ell^2$ norm)
		The $\ell^2$ norm of a scalar signal x, denoted $\|x\|_{\ell^2}$, is defined as follows:
		\begin{equation} \label{l2}
				\|x\|_{\ell^2}=\sqrt{\sum_{k=0}^{+\infty}x(k)^2}
		\end{equation}
		This norm exists only if $x$ is square-sommable, that is, if and only if the equation \ref{l2} converges.
	\end{thdef}

	\begin{thdef}\label{l_inf} ($\ell^\infty$ norm)
		The $\ell^\infty$ norm of a scalar signal x, denoted $\|x\|_{\ell^\infty}$, is the smallest upper bound among all values (absolute values) possible for the signal x, that is:
		\begin{equation} \label{l3}
				\|x\|_{\ell^\infty}=\sup_{k\in \mathbb{N}}|x(k)|
		\end{equation}
	\end{thdef}

	\subsubsection{Impulse response}
		\begin{thdef}\label{impresp} (Impulse Response)
			A SISO filter may be defined by it's impulse response, denoted $h$.
			$h$ is the impulse response of $\mathcal{H}$ to the impulsion of Dirac:
			\begin{equation} \label{ir}
					\delta(k)=
					\begin{cases}
						1 & when\  k=0 \\
						0 & else
					\end{cases}
			\end{equation}
			Indeed each input can be described as a sum of Dirac impulsions:
			\begin{equation} \label{dri}
				u = \sum_{l \geq 0}u(l)\delta_l
			\end{equation}
			where $\delta_l$ is a Dirac impulsion centered in l, that is:
			\begin{equation} \label{dirc}
					\delta(k)=
					\begin{cases}
						1 & when\  k=l \\
						0 & else
					\end{cases}
			\end{equation}
			The linearity condition of $\mathcal{H}$ implies:
			$\mathcal{H}(u)=\sum_{l \geq 0}u(l)\mathcal{H}(\delta_l)$.
			Time invariance gives: $\mathcal{H}(\delta_l)(k)=h(k-l)$.

			Then:
			\begin{equation} \label{convo}
				y(k) = \sum_{l \geq 0}u(l)h(k-l)= \sum_{l=0}^{k}u(k)h(k-l)
			\end{equation}

			This corresponds with the convolution product definition of $u$ by $h$,
			denoted $y= h * u$.

			Dealing with MIMO filters, we have $\boldsymbol{h} \in \mathbb{R}^{n_y \times n_u}$ as the impulse response of $\mathcal{H}$.
			$\boldsymbol{h}_{i,j}$ is the response on the $i$th output to the Dirac implusion on th $j$th input.

			The precedent equation becomes:
			\begin{equation} \label{convomimo}
					y_i(k) = \sum_{j=1}^{n_u}\sum_{l \geq 0}^{k}u_j(l)h_{i,j}(k-l) \hspace{30pt} \forall 1\leq i \leq n_y
			\end{equation}


				
		\end{thdef}
