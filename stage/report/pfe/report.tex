\documentclass[twoside]{article}
%\usepackage[a4paper,includeall,bindingoffset=0cm,margin=2cm,
%	            marginparsep=0cm,marginparwidth=0cm]{geometry}
\usepackage[a4paper,includeall]{geometry}
%\addtolength{\textheight}{-1cm}
\usepackage[english]{babel}
\usepackage[babel=true]{csquotes}
\usepackage[utf8]{inputenc}
\usepackage{pdfpages}
\usepackage[]{fancyhdr}
%\pagestyle{fancy}
%\setlength{\headheight}{50pt}
%\setlength{\headsep}{40pt}
%\fancyhead{}
%\lhead[\thesection]{\hspace{10pt}}
%\fancyhead[RO]{\rightmark}
%\fancyhead[LE]{\leftmark}
%\usepackage{amssymb}
%\setlength{\oddsidemargin}{1.575in}
%\setlength{\evensidemargin}{1.575in}
\setlength{\oddsidemargin}{1.18in}
\setlength{\hoffset}{-1in}
\setlength{\textwidth}{5.91in}
\setlength{\evensidemargin}{1.18in}
\setlength{\marginparwidth}{0pt}
\setlength{\marginparsep}{0pt}

\setlength{\topmargin}{1.18in}
\setlength{\voffset}{-1in}
\setlength{\textheight}{8.40in}

\pagenumbering{gobble} %TODO: uncomment this at the end of development

\usepackage[toc,page]{appendix}

\usepackage{amsthm}
\usepackage{amsfonts}
\usepackage{mathrsfs}
\usepackage{amssymb}
\usepackage{amsmath}
\usepackage[]{algorithm2e}


\usepackage{hyperref}
\hypersetup{
	colorlinks=true,
	breaklinks=true,
	linkcolor=blue,
	urlcolor=blue}
\newtheorem{thdef}{Definition}
\theoremstyle{remark}
\newtheorem{rk}{Remark}
\newtheorem{proposition}{Proposition}
\newtheorem{corollary}{Corollary}
\newcommand{\thref}[2]{\hyperref[#2]{#1 \ref*{#2}}}
\numberwithin{equation}{subsection}

\usepackage{url}
\usepackage[]{xcolor}
\definecolor{light-gray}{gray}{0.6}
\definecolor{blue}{RGB}{0,0,153}
\definecolor{blue1}{RGB}{0,102,204}
\definecolor{blue2}{RGB}{51,153,255}
\definecolor{blue3}{RGB}{0,153,153}
\definecolor{blue4}{RGB}{102,102,255}
\definecolor{red1}{RGB}{255,0,0}
\definecolor{orange1}{RGB}{255,153,51}
\definecolor{green}{RGB}{0,204,0}
\definecolor{purple}{RGB}{204,0,204}
\usepackage{graphicx}
\usepackage{multirow}
\usepackage{boxedminipage}
\usepackage{tikz}
\usetikzlibrary{calc} 

\setcounter{MaxMatrixCols}{20}

\tikzset{
  x=1ex,y=1ex,
  hwblock/.style={draw, rectangle, rounded corners=.3, very thick, fill=black!5, font=\sf, minimum height=5ex},
  hwbus/.style={very thick,>=stealth},
  hwwire/.style={thin, >=stealth, },
  hwword/.style={draw, rectangle, minimum height=3ex},
  bitwidth/.style={font=\scriptsize,midway,right}
}

\newcommand{\msbout}{m_{\text{out}}}
\newcommand{\appr}[1]{\widetilde{#1}}
\newcommand{\yout}{\widetilde{y}_{\text{out}}}
\newcommand{\abserr}{\varepsilon}
\newcommand{\epssopc}{\abserr_{\text{r}}}
\newcommand{\epsfinalround}{\abserr_{\text{f}}}
\newcommand{\TODO}{\textbf{TODO}}

\title{\vspace{-40pt}
	\fontsize{14}{16.8}\selectfont
	\textbf{Architecture synthesis for linear time-invariant filters}}
\author{\\ \fontsize{10}{12}\selectfont Antoine Martinet \\\\ \fontsize{9}{10.8}\selectfont Département Informatique \\ \fontsize{9}{10.8}\selectfont INSA de Lyon \\ \fontsize{9}{10.8}\selectfont 2014/2015}
%\date{2 February - 31 Jully, \\ \vspace{5pt} 2015}
\date{}

\begin{document}

\maketitle
%\tableofcontents
\noindent Under the responsability of : \\
Florent de Dinechin : INRIA Rhône-Alpes \\
Florent de Dinechin : Département Informatique
\vspace{20pt}

\begingroup

	\fontsize{9}{10.8}\selectfont
	\leftskip0.395in
	\rightskip\leftskip
	\noindent \textbf{Abstract.}
	The literature offers papers comparing different realizations of the same filter,
	and observing that more accuracy can be obtained at the expense of more multipliers.
	Such studies are relevant in software implementations, where the multiplier size is a given.
	In the context of hardware or FPGA implementations, more accuracy can also be bought by larger multipliers.
	This internship will study this trade-off and design tools that explore it.
	The ultimate objective is to generate the smallest or fastest filter satisfying a given accuracy specification.

	\vspace{20pt}

\endgroup
	
	\noindent \textbf{Key-words:}
	signal processing,
	architecture generation,
	LTI Filters.


\section{ Introduction }
	%\addcontentsline{toc}{section}{Introduction}

	Filters are nowadays essential tools for designing responsive systems.
	%As they appear in both signal processing and command communities,
	In signal processing, filters are used in
	radio signal coding and decoding, image and sound processing, and so on.
	In addition, they are also used in many control applications.


%	Although most of all the potential filters can be computed in software,
%	some interesting applications in hardware could help to accelerate such computations, 

	For most applications, filters can be computed in software.
	However, for performance reasons hardware implementations are needed in many cases.
	There is an interest in FPGAs implementations because they can help for prototyping a software defined radio (SDR).
	%interest on applications FPGAs implementations because help for prototyping software defined radio. SDR is the main researche subjevt of the SOCRATE team 
	%\TODO: make a sentence with comment above

	
	SDR is the main research subject of the SOCRATE team, at the CITI lab, where this work has been conducted.
	The goal of this internship was to design a parametric architecture generator for Linear Time Invariant (LTI) filters.
	This work has been integrated to FloPoCo tool,
	whose purpose is to generate architecture cores computing just right.


	This report first presents the context of LTI filters, SIF and their implementation in part \ref{Part1}.
	Part \ref{Part2} will then describe the details of the implementation brought in arbitrary precision.
	Finally, an overview of the future work will be presented in part \ref{Part3}.

\section{Tools for expressing filters and signal processing}
\label{Part1}
	\subsection[Fundamentals about signal processing]{Fundamentals about signal processsing\footnote{For a more complete introduction to the domain, please see appendix \ref{norem}} }
	\subsubsection{Definition of a signal}
	Lopez's PhD \cite{lopez} gives a good presentation of the state of the art.
	Many notations and definitions are kept from this PhD.

	Note that a bold symbol (for example $\boldsymbol{h}$) denotes a matrix, that may be a vector.
	On the contrary, a normal symbol (for example $\varepsilon_{t_1}(k)$) denotes a single value

	\begin{thdef}\label{sig} (Signal)
		Generally, a signal is a temporal variable, which takes a value from $\mathbb{R}$ at each time t.
		$x(t)$ denotes the value of the signal $x$ at the instant $t$.
		When dealing with discrete time events, the time will be represented by $k$.
		The notation will then be $x(k)$, which is said to be a sample.
		$\{x(k)\}_{k \geq 0}$ denotes all the values possible for the signal x.
		The rest of this report adresses vectors of signals $\textbf{x}$, where $\textbf{x}(k) \in \mathbb{R}^{n}$
	\end{thdef}

	%\subsubsection{$l^\infty$ norm}
	%\begin{thdef}\label{l_inf} ($\ell^\infty$ norm)
	%	The $\ell^\infty$ norm of a scalar signal x, denoted $\|x\|_{\ell^\infty}$, is the smallest upper bound among all values (absolute values) possible for the signal x, that is:
	%	\begin{equation} \label{l3}
	%			\|x\|_{\ell^\infty}=\sup_{k\in \mathbb{N}}|x(k)|
	%	\end{equation}
	%\end{thdef}

	\subsubsection[Linear Time Invariant Filters (LTI Filters)]{Linear Time Invariant Filters (LTI filters)\footnote{For more information about the studied filters, please refer to appendix \ref{fildef}}}
	A filter, denoted by its transfer function $\mathcal{H}$, is an application which transforms a signal vector $\boldsymbol{u}$ (with $dim(\boldsymbol{u}) = n_u$ )
	into a signal vector $\boldsymbol{y} = \mathcal{H}(\boldsymbol{u})$, of size $dim(\boldsymbol{y}) = n_y$ . The case where $n_u = n_y = 1$ is referred as Single Input Single Output
	(SISO) filters. Other cases are referred as Multiple Input Multiple Output (MIMO) filters.

	\begin{thdef} (Linear Time Invariant Filter)

		Linearity:
		$$ \mathcal{H}(\alpha \cdot \boldsymbol{u}_1+ \beta \cdot \boldsymbol{u}_2)= \alpha\cdot\mathcal{H}(\boldsymbol{u}_1) +  \beta\cdot\mathcal{H}(\boldsymbol{u}_2)$$

		Time invariance:
		$$ \{\mathcal{H}(\boldsymbol{u})(k-k_0)\}_{k\geq0} = \mathcal{H}(\{\boldsymbol{u}(k-k_0)\}_{k \geq 0} ) $$
	\end{thdef}

	%\subsubsection{Impulse response}
	%\begin{thdef} (Impulse Response)
	%A SISO filter may be defined by its impulse response, denoted $h$. $h$ is the
	%impulse response of H to the impulsion of Dirac.
	%Indeed each input signal can be described as a sum of Dirac impulsions:
	%$$u=\sum_{i\geq0}u(l)\delta_l$$
	%where $\delta_l$ is a Dirac impulsion centered in $l$, that is:
	%\begin{equation}
	%	\delta_l(k) =
	%	\begin{cases}
	%		1 & \hspace{5pt} when \hspace{5pt} k=l\\
	%		0 & \hspace{5pt} else\\
	%	\end{cases}
	%\end{equation}
	%The linearity condition of $\mathcal{H}$ implies: $\mathcal{H}(u) = \sum_{l\geq0}u(l)\mathcal{H}(\delta_l)$.
	%Time invariance gives: $\mathcal{H}(\delta_l)(k)=h(k-l)$.
	%Then the computation from inputs to outputs takes this form:
	%$$y(k)=\sum_{l\geq0}u(l)h(k-l)=\sum_{l=0}^ku(k)h(k-l)$$
	%This corresponds with the convolution product definition of $u$ by $h$, denoted $y = h * u$.
	%%Dealing with MIMO filters, we have $\boldsymbol{h} \in \mathbb{R}^{n_y \times n_u}$ as the impulse response of $\mathcal{H}$. $\boldsymbol{h}_{i,j}$ is the response on the
	%Dealing with MIMO filters, $\boldsymbol{h}(k) \in \mathbb{R}^{n_y \times n_u}$ is the impulse response of $\mathcal{H}$. $\boldsymbol{h}_{i,j}(k)$ is the response on the
	%ith output to the Dirac implusion on the j-th input.
	%The precedent equation becomes:
	%$$y_i(k)=\sum_{j=1}^{n_u}\sum_{l=0}^ku_j(l)h_{i,j}(k-l), \hspace{5pt} \forall 1 \leq i \leq n_y$$
	%
	%\end{thdef} 

	\subsubsection{Worst-Case Peak Gain (WCPG) of a Filter: the maximum possible amplification}
	\begin{thdef} (Worst-Case Peak Gain)
		The worst case peak gain is defined as the maximum amplification
		possible over all potential inputs through the filter.
		$$\|\mathcal{H}\|_{wcpg}=\sup_{u\neq0}\frac{\|h*u\|_{l^{\infty}}}{\|u\|_{l^{\infty}}}$$
		with $h$ the impulse response of $\mathcal{H}$, $u$ the input signal, and $h * u$ the convolution product of $h$ by $u$ (output of the
				filter).
	
	\end{thdef}
%	\subsection{FIR and IIR: two filters families}
%	There are two types of LTI filters: \textit{Finite impulse response} (FIR) and \textit{Infinite impulse response} (IIR) filters.
%	Formally, the impulse response is finite when:
%	\begin{equation} \label{finimp}
%		\exists n \in \mathbb{N} | \forall k \geq n, h(k)=0
%	\end{equation}
%	The smallest $n$ verifying \ref{finimp} is referred as the order of the filter. So a $n$-order FIR can be described by the
%	following equation:
%	\begin{equation} \label{firdef}
%		y(k)=\sum_{i=0}^n b_i u(k-i)
%	\end{equation}
%
%	An IIR will be described as following:
%	\begin{equation} \label{iirdef}
%		y(k)=\sum_{i=0}^n b_i u(k-i) - \sum_{i=0}^n a_i y(k-i)
%	\end{equation}
%
%	%Here one can observe that the output at time $k$ depends also on all previous $n$ outputs (loopback). 
%	%Also remark that a FIR can be seen as an IIR with $\forall i \in [0,n],a_i=0$
%	%The impulse response can then be deduced from \ref{iirdef} by resolving the recurrence relation:
%
%	%\begin{equation}
%	%	h(k) =
%	%	\begin{cases}
%	%		0 & when \hspace{5pt} k<l\\
%	%		b_k - \sum_{l=1}^n a_l h(k-l) & when \hspace{5pt} 0\leq k \leq n\\
%	%		\sum_{l=1}^n a_l h(k-l) & when \hspace{5pt} n< k\\
%	%	\end{cases}
%	%\end{equation}

	\subsection{Different realizations: how to compute the output of a filter?}
	\begin{thdef} (realization)
	A realization can be defined as an algorithm describing how to compute outputs
	from inputs. However, a realization does not describes the details of basic operations (format, size,
	rounding, etc...)
	\end{thdef}
	It is important to know that all realizations of a filter are mathematically equivalent to each other (infinite
	precision). But in finite precision, rounding aspects have a huge impact on the correctness of results.
	Most of the time in this report, realizations will be referred as forms.

%	\subsubsection{Direct and transposed forms}
%	Direct and transposed forms are classic realizations. A good description of these forms can be found in Lopez’ and Hilaire’s PhDs \cite{lopez} \cite{hilaire}. 
%	The direct form has been implemented in the FoPoCo project,
%	The hardware implementation of anIIR can be seen on figure \ref{fig:ltiarch}
%
%\begin{figure*}[h]
%  \centering
%  \begin{tikzpicture}
%     \draw[dotted,black, fill=yellow!20] (-5ex,-3.5ex) rectangle +(69ex,-15ex);   
%     \node[black]  at (-2ex,-17.5ex) {{\small SOPC}};   
%    \draw[hwbus] (-8, 0) node[left] {$u(k)$} --  ++(8, 0)  ;
%    \draw[hwbus,->] (-8, 0) --  ++(3, 0); % just for the arrow
%    \foreach \i in {0,...,3} {
%      \draw[hwbus, ->] ($(8*\i, 0)$) --  ++(0, -5);
%      \draw[hwblock] ($(8*\i, -5)$) -- ++(3, 0) -- ++(-3, -4) -- ++(-3, 4) -- cycle; 
%      \node (n) at ($(8*\i , -6.5)$)  {$b_{\i}$} ;
%%      \draw ($(8*\i  + 0.5, -11)$) node[left,tt=black!50] {\footnotesize $p_b(k,\i)$} ;
%%      \draw ($(8*\i  - 0.2, -11)$) node[right,text=black!50] {\footnotesize $p_b(k,\i)$} ;
%
%%      \draw[hwbus, ->] ($(8*\i ex, 0ex)$) --  ++(0, -5ex);
%    }
%
%    
%    \draw[hwbus] (0, -9) -- ++(0,-6);
%      \coordinate (n) at  (0,-15);
%    \foreach \i in {1,...,3} {
%      \draw[line width=3pt] ($(8*\i  - 4, -3)$) --  +(0, 6);
%      \draw[hwbus] ($(8*\i  - 8, 0)$) --  +(8,0) node [above,text=blue] {\footnotesize $u(k-\i)$};
%      \draw[hwbus,->] ($(8*\i  - 8, 0)$) --  ++(3, 0); % just for the arrow
%      % The adders 
%      \coordinate (nm1) at  (n.east);
%      \draw ($(8*\i , -15)$) node[hwblock,circle,minimum height=3] (n) {$+$};
%      \draw[hwbus, ->]  ($(8*\i , -9)$) -- (n.north);
%      \draw[hwbus, ->]  (nm1) -- (n.west);
%      \draw[hwbus]  (n.east) -- ++ (3,0);
%    }
%
%    \draw (74, -15) node[hwblock,align=center] (fr) {final\\round} ;
%    \draw[hwbus, <-] (fr.west) -- ++(-15,0) node [near end] {/} node [near end,below] {\footnotesize$(\msbout, p-g$)} node[near start,above] {$\appr{y}(k)$} -- ++(-1,0) ;
%    \draw[hwbus, ->] (fr.east) -- ++(8,0) node [midway] {/} node [midway,below] {\footnotesize$(\msbout,p)$} node[right] {$\yout(k)$} ;
%
%    \foreach \i in {1,...,3} {
%      \draw[hwbus, ->] ($(-8*\i  + 60 , 0)$) --  ++(0, -5);
%      \draw[hwblock] ($(-8*\i  + 60 , -5)$) -- ++(3, 0) -- ++(-3, -4) -- ++(-3, 4) -- cycle; 
%      \node (ai) at ($(-8*\i  + 60 , -6.5)$)  {$a_{\i}$} ;
%      %\draw ($(-8*\i  +60 + 0.5, -11)$) node[left,text=black!50] {\footnotesize $p_a(k,\i)$} ;
%      %\draw ($(-8*\i  +60 - 0.2, -11)$) node[right,text=black!50] {\footnotesize $p_a(k,\i)$} ;
%      \draw ($(-8*\i  + 60 , -15)$) node[hwblock,circle,minimum height=3] (n) {$+$};
%      \draw (n.north) node[left]{\bf -};
%      \draw[hwbus, ->] ($(-8*\i  + 60, -9)$) --  (n.north);
%      \draw[hwbus, <-] (n.west) -- ++(-5,0);
%      % The registers
%      \draw[hwbus] ($(-8*\i  + 60  +8, 0)$) --  ++(-8, 0) node [above,text=blue] {\footnotesize $\appr{y}(k-\i)$};
%      \draw[hwbus,->] ($(-8*\i  + 60  +8, 0)$) --  ++(-3, 0); % just for the arrow
%      \draw[line width=3pt] ($(-8*\i  +60 + 4, -3)$) --  +(0, 6);
%    }
%    \draw[hwbus] ($(60 , 0)$) --  ++(0,-15);
%    \draw[hwbus,<-] ($(60 , -5)$) --  ++(0,-5);
%
%
%  \end{tikzpicture}
%
%\caption{Abstract architecture for the direct form realization of an LTI filter \label{fig:ltiarch}}
%\end{figure*}
%
%	\subsubsection{State-space representation: a recurrence to define an infinite response}
	%This type of realization consists in expressing the evolution of a system considering its state at time $k$. In
	\vspace{10pt}
	The state-space representation is a type of realization consists in expressing the evolution of a system considering its state at time $k$. In
	continuous time, it is described by differential equations at first order. In discret time (in which we
	are interested in), it is described by a simple recurrence:
	\begin{equation} \label{abcddef}
		\begin{cases}
			\boldsymbol{x}(k+1)= \boldsymbol{Ax}(k) + \boldsymbol{Bu}(k) \\
			\boldsymbol{y}(k+1)= \boldsymbol{Cx}(k) + \boldsymbol{Du}(k)
		\end{cases}
	\end{equation}

	Where $\boldsymbol{x}(k) \in \mathbb{R}^{n_x}$ is the state vector,
	$\boldsymbol{u}(k) \in \mathbb{R}^{n_u}$ is the input vector and
	$\boldsymbol{y}(k) \in \mathbb{R}^{n_y}$ is the output vector, at time k.
	The matrices $\boldsymbol{A} \in \mathbb{R}^{n_x \times n_x}$ , $\boldsymbol{B} \in \mathbb{R}^{n_x \times n_u}$,
	$\boldsymbol{C} \in \mathbb{R}^{n_y \times n_x}$, and $\boldsymbol{D} \in \mathbb{R}^{n_y \times n_u}$,
	with $\boldsymbol{x}(0)$ are sufficient to describe an LTI filter, with convention $\boldsymbol{x}(k)=\boldsymbol{u}(k)=0 \hspace{5pt}, \hspace{5pt} \forall k<0$.

	\subsection{The Specialized Implicit Form (SIF): a unified representation}
	\subsubsection{Definition}
	The classical state-space representation is intuitive, but it doesn't take into account the reality of implementation.
	The specialized implicit form (SIF) was introduced in \cite{sifd}, is well detailed in Hilaire's PhD and associated papers \cite{hilaire,sif},
	and is a good answer to this problem.
	Indeed, dealing with finite precision and error amplification, the order in which operations are done becomes crucial.
	Another motivation is to have a unique representation for any realization of LTI filters,
	that allows to compute every degradation measures instead of redevelopping them for each new realization.
	This form distinguishes computations done at one time from computations done the other times. As well as
	in a state-space, $\boldsymbol{x}$-coordinates are state variables, but in addition to that, $\boldsymbol{t}$-coordinates are intermediate variables.
	The SIF is described as following:
	\begin{equation} \label{sifdef}
		\begin{pmatrix}
			\boldsymbol{J} & \boldsymbol{0} & \boldsymbol{0} \\
			\boldsymbol{-K} & \boldsymbol{I}_{n_x} & \boldsymbol{0} \\
			\boldsymbol{-L} & \boldsymbol{0} & \boldsymbol{I}_{n_y} 
		\end{pmatrix}
		\begin{pmatrix}
			\boldsymbol{t} (k+1)  \\
			\boldsymbol{x} (k+1)  \\
			\boldsymbol{y} (k) 
		\end{pmatrix}
		=
		\begin{pmatrix}
			\boldsymbol{0} & \boldsymbol{M} & \boldsymbol{N} \\
			\boldsymbol{0} & \boldsymbol{P} & \boldsymbol{Q} \\
			\boldsymbol{0} & \boldsymbol{R} & \boldsymbol{S} 
		\end{pmatrix}
		\begin{pmatrix}
			\boldsymbol{t} (k)  \\
			\boldsymbol{x} (k)  \\
			\boldsymbol{u} (k) 
		\end{pmatrix}
	\end{equation}
%	With $n_t$, $n_x$, $n_y$ and $n_u$ the sizes of $\boldsymbol{t}$, $\boldsymbol{x}$, $\boldsymbol{y}$ and $\boldsymbol{u}$, respectively.
%	$\boldsymbol{J}$, is a lower triangular matrix, with
%	diagonal entries equal to 1.
%	%Then we have the following dimensions for the previous matrices:
%	The previous matrices have the following dimensions:
%
%	\begin{eqnarray}
%		\boldsymbol{J} \in \mathbb{R}^{n_t \times n_t},\boldsymbol{M} \in \mathbb{R}^{n_t \times n_x},\boldsymbol{N} \in \mathbb{R}^{n_t \times n_u}, \nonumber \\
%		\boldsymbol{K} \in \mathbb{R}^{n_x \times n_t},\boldsymbol{P} \in \mathbb{R}^{n_x \times n_x},\boldsymbol{Q} \in \mathbb{R}^{n_x \times n_u}, \\
%		\boldsymbol{L} \in \mathbb{R}^{n_y \times n_t},\boldsymbol{R} \in \mathbb{R}^{n_y \times n_x},\boldsymbol{S} \in \mathbb{R}^{n_y \times n_u}, \nonumber \\
%	\end{eqnarray}

	The best way to understand the SIF may be to see it as an algorithm, each line of the equation \ref{sifdef} corresponding to a sequential step of the computation.
	%The algorithm results as follows:
	%\begin{algorithm}
	%	\For{int i = 0 ; $i \leq n_t$; i++}{
	%		$\boldsymbol{t}_i(k+1) \leftarrow - \sum\limits\limits_{j<i} \boldsymbol{J}_{ij}\boldsymbol{t}_j(k+1) + \sum\limits_{j=1}^{n_x} \boldsymbol{M}_{ij}\boldsymbol{x}_j(k) + \sum\limits_{j=1}^{n_u} \boldsymbol{N}_{ij}\boldsymbol{u}_j(k)$
	%	}
	%	\For{int i = 0 ; $i \leq n_x$; i++}{
	%		$\boldsymbol{x}_i(k+1) \leftarrow - \sum\limits_{j=1}^{n_t} \boldsymbol{K}_{ij}\boldsymbol{t}_j(k+1) + \sum\limits_{j=1}^{n_x} \boldsymbol{P}_{ij}\boldsymbol{x}_j(k) + \sum\limits_{j=1}^{n_u} \boldsymbol{Q}_{ij}\boldsymbol{u}_j(k)$
	%	}
	%	\For{int i = 0 ; $i \leq ny$; i++}{
	%		$\boldsymbol{y}_i(k) \leftarrow - \sum\limits_{j=1}^{n_t} \boldsymbol{L}_{ij}\boldsymbol{t}_j(k+1) + \sum\limits_{j=1}^{n_x} \boldsymbol{R}_{ij}\boldsymbol{x}_j(k) + \sum\limits_{j=1}^{n_u} \boldsymbol{S}_{ij}\boldsymbol{u}_j(k)$
	%	}
	%	\caption{Computation of SIF outputs from inputs}
	%\end{algorithm}

	%Here, it is important to see that the form of $\boldsymbol{J}$ allows to compute the $\boldsymbol{t}_i$ sequentially. The algorithm can
	%then be described as follows:
	%\begin{algorithm} \label{algomat}
	%	\For{int i = 0 ; $i \leq n_t$; i++}{
	%		$\boldsymbol{t}_i(k+1) \leftarrow - \boldsymbol{J'}_{i}\boldsymbol{t}(k+1) + \boldsymbol{M}_{i}\boldsymbol{x}(k) +\boldsymbol{N}_{i}\boldsymbol{u}(k)$
	%	}
	%	\For{int i = 0 ; $i \leq n_x$; i++}{
	%		$\boldsymbol{x}_i(k+1) \leftarrow - \boldsymbol{K}_{i}\boldsymbol{t}(k+1) + \boldsymbol{P}_{i}\boldsymbol{x}(k) +\boldsymbol{Q}_{i}\boldsymbol{u}(k)$
	%	}
	%	\For{int i = 0 ; $i \leq ny$; i++}{
	%		$\boldsymbol{y}_i(k) \leftarrow - \boldsymbol{L}_{i}\boldsymbol{t}(k+1) + \boldsymbol{R}_{i}\boldsymbol{x}(k) + \boldsymbol{S}_{i}\boldsymbol{u}(k)$
	%	}
	%	\caption{Simplified matricial algorithm}
	%\end{algorithm}

	%With $\boldsymbol{J'} = \boldsymbol{J} - I_{n_t}$.

	Values of the vector $\boldsymbol{t}(k+1)$ are computed and used at the same iterations, so they are not kept in memory.
	As the equation \ref{sifdef} is mostly full of zeros, it is more convenient to use it’s compressed formulation, which is denoted
	$\boldsymbol{Z}$:
	
	\begin{equation} \label{zmatrix}
		\boldsymbol{Z}=
		\begin{pmatrix}
			\boldsymbol{-J} & \boldsymbol{M} & \boldsymbol{N} \\
			\boldsymbol{K} & \boldsymbol{P} & \boldsymbol{Q} \\
			\boldsymbol{L} & \boldsymbol{R} & \boldsymbol{S} 
		\end{pmatrix}
	\end{equation}

	The community usually takes $-\boldsymbol{J}$ for simplicity within further computations.

	%The SIF can of course be transformed into an equivalent ABCD (classic state-space) form, which gives:
	The SIF can of course be transformed into an equivalent ABCD (classic state-space) form.
	About parallelism, it is useful to remark that $t(k+1)$ are computed sequentially, while $x(k+1)$ and $y(k)$ can be computed in parallel.
%
%	\begin{eqnarray} \label{abcdtranspose}
%		\boldsymbol{A_Z} = \boldsymbol{KJ}^{-1}\boldsymbol{M} +\boldsymbol{P}, \hspace{5pt}
%		\boldsymbol{B_Z} = \boldsymbol{KJ}^{-1}\boldsymbol{N} +\boldsymbol{Q}, \nonumber \\
%		\boldsymbol{C_Z} = \boldsymbol{LJ}^{-1}\boldsymbol{M} +\boldsymbol{R}, \hspace{5pt}
%		\boldsymbol{D_Z} = \boldsymbol{LJ}^{-1}\boldsymbol{N} +\boldsymbol{S}, \nonumber \\
%	\end{eqnarray}
%
%	with:
%	\begin{eqnarray}
%		\boldsymbol{A_Z} \in \mathbb{R}^{n_x \times n_x},
%		\boldsymbol{B_Z} \in \mathbb{R}^{n_x \times n_u}, \nonumber \\
%		\boldsymbol{C_Z} \in \mathbb{R}^{n_y \times n_x},
%		\boldsymbol{D_Z} \in \mathbb{R}^{n_y \times n_u},
%	\end{eqnarray}

	\subsubsection{Workflow for filter implementation}
		%Here, we won't discuss the optimality of the SIF description.
		The choice of the SIF realization is not concerned by this work.
		This choice is up to the user, who may have a lot of good reasons not to design a realization that seem intuitive for us.
		The choice of the realization is a job done at LIP6 in the PEQUAN team, under the metalibm ANR project.
		The present work is just a hardware backend for the choice part.
		Of course, optimizations are available at every level.
		The expected use of this work is shown on figure \ref{fig:workflow}.


		\begin{figure}[h] 
		  \centering
		  \begin{tikzpicture}[x=0.8cm,y=1cm]
			\draw (-9.5, 1.0) -- (-8,1.0) [->, thick] node[above,near start, text width = 3cm, align=center]{User filter specification};
			\draw (-3.5, 1.0) -- (0.0,1.0) [->, thick] node[below,xshift=-1.4cm]{$\begin{pmatrix}\boldsymbol{-J} & \boldsymbol{M} & \boldsymbol{M}_t \\ \boldsymbol{K} & \boldsymbol{P} & \boldsymbol{M}_x \\ \boldsymbol{L} & \boldsymbol{R} & \boldsymbol{M}_y \end{pmatrix}$};
			\draw (-8,-0.3) rectangle ++(4.5,2.4)[thick] node [midway, text width = 4cm, align=center]{Realization Choice \\ by \\ Hilaire, Volkova  \\ team PEQUAN, LIP6 \\(Front-End)}; 
			\draw (0.0,0.0) rectangle ++(3.0,2)[thick] node [midway, text width = 3cm, align=center]{Flopoco core generation \\ (this work) \\(Back-End)}; 

			\draw (4,0) rectangle ++(2.5,2)[thick] node [midway]{VHDL}; 
			\draw (3.0, 1) -- (4,1) [->, thick] node[above,near start]{};

		  \end{tikzpicture}

		\caption{Workflow overview of tools usage \label{fig:workflow}}
		\end{figure}










%	\subsection{Adjustments in arbitrary precision}

	





%		
%	\subsection{Canonical specification}
%		
%		The intuitive and first way to describe an LTI filter is to specify the output as a function of the inputs:
%		$$y(k)=\sum_{i=0}^n b_i u(k-i)-\sum_{i=1}^n a_i y(k-i)$$
%
%	\subsection{Z transform}
%	$$X(z)=\mathcal{Z}\{x\}=\sum_{k=0}^{+\infty} x(k) z^{-k}$$
%	
%	\subsection{Transfert Function}
%	A filter is usually described by it's transfert function, defined as:
%
%	$$H(z)=\frac{Y(z)}{U(z)}=\frac{\sum_{i=0}^n b_i z^{-i}}{ 1 + \sum_{i=1}^n a_i z^{-i}}, \;\;\;\; \forall z \in \mathbb{C}$$ 
%	\subsection{Impulse response}
%	$\delta$
%	\subsection{Worst case peak gain (WCPG)}
%	$$\| \mathcal{H} \|_{WCPG}=\sup_{i\neq 0} \frac{\|h*u\|_{l^\infty}}{\|u\|_{l^\infty}} $$
%
%	$$\| \mathcal{H} \|_{WCPG}= \sum_{k \geq 0} |h(k)| $$
%	\subsection{Realisations}



	





%\newpage
\section{Architecture generation}
\label{Part2}
\subsection{}

In SOPCs architectures, the accuracy is deducible from the inputs/outputs specifications and the size of the constants.

This is described in %\paper{SOPCs}.

Dealing with feedback inputs, the question of precision is more compliated.
Indeed, when results loop back to inputs, as soon as we are in finite precision, the error is amplified by a certain amount, depending on the coefficients, at each pass through the filter.

The main idea to deimension such filters is to consider the total error as a single filter.
The result of this filter is then added to the perfect filter to get the final output.

In finite precision, sizes are constrained to be all the same. The demonstration of the size computation has been described in Lopez' PHD.
The idea now is to see what we can do in arbitrary precision.

Here we have to compute each size at each step of the computation. Indeed, the WCPG is not useful for the first part of a FIR, as it has no loop.
So we just need the WCPG for the second part, because it is just in this part of the circuit that there is a potential error amplification.

Direct and transposed forms are not directly transposable into SIF, but this problem is secondary.

\subsection{Algorithm}
\begin{figure}[!h]
\begin{center}
\scalebox{6}{  %\begin{center}
\resizebox{!}{80pt}{
  \begin{tikzpicture}

    \normalsize 
    \node[draw, blue!100,   ultra thick, rectangle, minimum width = 3.4*34.1ex, minimum height=4.0ex] (sel1)  at (42.0em, 45.7em) {} ;
    \node[draw, blue!100,   ultra thick, rectangle, minimum width = 3.4*34.1ex, minimum height=4.0ex] (sel2)  at (42.0em, 43.5em) {} ;
    \node[draw, blue!100,   ultra thick, rectangle, minimum width = 3.4*34.1ex, minimum height=4.0ex] (sel3)  at (42.0em, 41.4em) {} ;
    \node[draw, blue!100,   ultra thick, rectangle, minimum width = 3.4*34.1ex, minimum height=4.0ex] (sel4)  at (42.0em, 39.3em) {} ;
    \node[draw, blue!100,   ultra thick, rectangle, minimum width = 3.4*34.1ex, minimum height=4.0ex] (sel5)  at (42.0em, 37.1em) {} ;
    \node[draw, green!100,  ultra thick, rectangle, minimum width = 3.4*34.1ex, minimum height=4.0ex] (sel6)  at (42.0em, 35.2em) {} ;
    \node[draw, green!100,  ultra thick, rectangle, minimum width = 3.4*34.1ex, minimum height=4.0ex] (sel7)  at (42.0em, 32.9em) {} ;
    \node[draw, green!100,  ultra thick, rectangle, minimum width = 3.4*34.1ex, minimum height=4.0ex] (sel8)  at (42.0em, 30.8em) {} ;
    \node[draw, green!100,  ultra thick, rectangle, minimum width = 3.4*34.1ex, minimum height=4.0ex] (sel9)  at (42.0em, 28.6em) {} ;
    \node[draw, green!100,  ultra thick, rectangle, minimum width = 3.4*34.1ex, minimum height=4.0ex] (sel10) at (42.0em, 26.4em) {} ;
    \node[draw, red!100,    ultra thick, rectangle, minimum width = 3.4*34.1ex, minimum height=4.0ex] (sel11) at (42.0em, 24.5em) {} ;

	\node at (28, 35em) { \Huge $Z = $};
	\normalsize
  \node at (98, 35em) {
	  \resizebox{!}{120pt}{
		$\begin{pmatrix}
			1       & 0       & 0       & 0       & 0       & m_{1,1} & 0       & 0       & 0       & 0       & 0       \\
			j_{2,1} & 1       & 0       & 0       & 0       & 0       & m_{2,2} & 0       & 0       & 0       & 0       \\
			j_{3,1} & j_{3,2} & 1       & 0       & 0       & 0       & 0       & m_{3,3} & 0       & 0       & 0       \\
			j_{4,1} & j_{4,2} & j_{4,3} & 1       & 0       & 0       & 0       & 0       & m_{4,4} & 0       & 0       \\
			j_{5,1} & j_{5,2} & j_{5,3} & j_{5,4} & 1       & 0       & 0       & 0       & 0       & m_{5,5} & 0       \\

			k_{1,1} & 1       & 0       & 0       & 0       & p_{1,1} & 0       & 0       & 0       & 0       & q_{1,1} \\
			k_{2,1} & 0       & 1       & 0       & 0       & 0       & p_{2,2} & 0       & 0       & 0       & q_{2,1} \\
			k_{3,1} & 0       & 0       & 1       & 0       & 0       & 0       & p_{3,3} & 0       & 0       & q_{3,1} \\
			k_{4,1} & 0       & 0       & 0       & 1       & 0       & 0       & 0       & p_{4,4} & 0       & q_{4,1} \\
			k_{5,1} & 0       & 0       & 0       & 0       & 0       & 0       & 0       & 0       & p_{5,5} & q_{5,1} \\

			1	    & 0       & 0       & 0       & 0       & 0       & 0       & 0       & 0       & 0       & 0       \\
		\end{pmatrix}$
	  }
  };

%	$\begin{pmatrix}
%		1       & 0       & 0       & 0       & 0       & m_{1,1} & m_{1,2} & m_{1,3} & m_{1,4} & m_{1,5} & n_{1,1} \\
%		j_{2,1} & 1       & 0       & 0       & 0       & m_{2,1} & m_{2,2} & m_{2,3} & m_{2,4} & m_{2,5} & n_{2,1} \\
%		j_{3,1} & j_{3,2} & 1       & 0       & 0       & m_{3,1} & m_{3,2} & m_{3,3} & m_{3,4} & m_{3,5} & n_{3,1} \\
%		j_{4,1} & j_{4,2} & j_{4,3} & 1       & 0       & m_{4,1} & m_{4,2} & m_{4,3} & m_{4,4} & m_{4,5} & n_{4,1} \\
%		j_{5,1} & j_{5,2} & j_{5,3} & j_{5,4} & 1       & m_{5,1} & m_{5,2} & m_{5,3} & m_{5,4} & m_{5,5} & n_{5,1} \\
%
%		k_{1,1} & k_{1,1} & k_{1,1} & k_{1,1} & k_{1,1} & p_{1,1} & p_{1,1} & p_{1,1} & p_{1,1} & p_{1,1} & q_{1,1} \\
%		k_{2,1} & k_{2,1} & k_{2,1} & k_{2,1} & k_{2,1} & p_{1,1} & p_{1,1} & p_{1,1} & p_{1,1} & p_{1,1} & q_{2,1} \\
%		k_{3,1} & k_{3,1} & k_{3,1} & k_{3,1} & k_{3,1} & p_{1,1} & p_{1,1} & p_{1,1} & p_{1,1} & p_{1,1} & q_{3,1} \\
%		k_{4,1} & k_{4,1} & k_{4,1} & k_{4,1} & k_{4,1} & p_{1,1} & p_{1,1} & p_{1,1} & p_{1,1} & p_{1,1} & q_{4,1} \\
%		k_{5,1} & k_{5,1} & k_{5,1} & k_{5,1} & k_{5,1} & p_{1,1} & p_{1,1} & p_{1,1} & p_{1,1} & p_{1,1} & q_{5,1} \\
%                                                       
%		l_{5,1} & l_{5,1} & l_{5,1} & l_{5,1} & l_{5,1} & r_{1,1} & r_{1,2} & r_{1,3} & r_{1,4} & r_{1,5} & s_{1,1} \\
%	\end{pmatrix}$

		\node[draw, blue!100, thick, rectangle, minimum width = 3.4*5ex, minimum height=4ex] (d1) at (5em, 13em) {} ;
		  \node (n1) at (5em, 13em)  {\Large $m_{1,1}$} ;
		  \draw[black!25, very thin] ($(n1)-(3.7ex,-1.5ex)$) -- ++(0, -3ex);
		  \draw[black!25, very thin] ($(n1)+(3.7ex, 1.5ex)$) -- ++(0, -3ex);

		\node[draw, blue!100, thick, rectangle, minimum width = 3.4*4.4ex, minimum height=4ex] (d2) at (9.6em,9em) {} ;
		  \node (n2) at (8em, 9em)  {\large $-j_{2,1}$} ;
		  \node (n3) at (11.2em, 9em)  {\Large $m_{2,2}$} ;
		  \draw[black!25, very thin] ($(n2)+(3.7ex, 1.5ex)$) -- ++(0, -3ex);

		\node[draw, blue!100, thick, rectangle, minimum width = 3.4*6.5ex, minimum height=4ex] (d3) at (15.2em,5em) {} ;
		  \node (n4) at (12em, 5em)    {\large $-j_{3,1}$} ;
		  \node (n5) at (15.2em, 5em)  {\large $-j_{3,2}$} ;
		  \node (n6) at (18.2em, 5em)  {\Large $m_{3,3}$} ;
		  \draw[black!25, very thin] ($(n4)+(3.7ex, 1.5ex)$) -- ++(0, -3ex);
		  \draw[black!25, very thin] ($(n5)+(3.7ex,1.5ex)$) -- ++(0, -3ex);

		\node[draw, blue!100, thick, rectangle, minimum width = 3.4*8.5ex, minimum height=4ex] (d4) at (24.6em,-1em) {} ;
		  \node (n7) at (20em, -1em)     {\large $-j_{4,1}$} ;
		  \node (n8) at (23.2em, -1em)   {\large $-j_{4,2}$} ;
		  \node (n9) at (26.2em, -1em)   {\large $-j_{4,3}$} ;
		  \node (n10) at (29.2em, -1em)  {\Large $m_{4,4}$} ;
		  \draw[black!25, very thin] ($(n7)+(3.7ex, 1.5ex)$) -- ++(0, -3ex);
		  \draw[black!25, very thin] ($(n8)+(3.7ex,1.5ex)$) -- ++(0, -3ex);
		  \draw[black!25, very thin] ($(n9)+(3.7ex,1.5ex)$) -- ++(0, -3ex);

		\node[draw, blue!100, thick, rectangle, minimum width = 3.4*10.6ex, minimum height=4ex] (d5) at (34.2em, -8em) {} ;
		  \node (n11) at (28em, -8em)    {\large $-j_{5,1}$} ;
		  \node (n12) at (31.2em, -8em)  {\large $-j_{5,2}$} ;
		  \node (n13) at (34.2em, -8em)  {\large $-j_{5,3}$} ;
		  \node (n14) at (37.2em, -8em)  {\large $-j_{5,4}$} ;
		  \node (n15) at (40.2em, -8em)  {\Large $m_{5,5}$} ;
		  \draw[black!25, very thin] ($(n11)+(3.7ex, 1.5ex)$) -- ++(0, -3ex);
		  \draw[black!25, very thin] ($(n12)+(3.7ex,1.5ex)$) -- ++(0, -3ex);
		  \draw[black!25, very thin] ($(n13)+(3.7ex,1.5ex)$) -- ++(0, -3ex);
		  \draw[black!25, very thin] ($(n14)+(3.7ex,1.5ex)$) -- ++(0, -3ex);

		\node[draw, green!100, thick, rectangle, minimum width = 3.4*8.5ex, minimum height=4ex] (d6) at (41.7em, -16em) {} ;
		  \node (n16) at (37em, -16em)    {\Large $k_{1,1}$} ;
		  \node (n17) at (40.2em, -16em)  {\Large $1$} ;
		  \node (n18) at (43.2em, -16em)  {\Large $p_{1,1}$} ;
		  \node (n19) at (46.2em, -16em)  {\Large $q_{1,1}$} ;
		  \draw[black!25, very thin] ($(n16)+(3.7ex, 1.5ex)$) -- ++(0, -3ex);
		  \draw[black!25, very thin] ($(n17)+(3.7ex,1.5ex)$) -- ++(0, -3ex);
		  \draw[black!25, very thin] ($(n18)+(3.7ex,1.5ex)$) -- ++(0, -3ex);

		\node[draw, green!100, thick, rectangle, minimum width = 3.4*8.5ex, minimum height=4ex] (d7) at (46.7em, -23em) {} ;
		  \node (n20) at (42em, -23em)    {\Large $k_{2,1}$} ;
		  \node (n21) at (45.2em, -23em)  {\Large $1$} ;
		  \node (n22) at (48.2em, -23em)  {\Large $p_{2,2}$} ;
		  \node (n23) at (51.2em, -23em)  {\Large $q_{2,1}$} ;
		  \draw[black!25, very thin] ($(n20)+(3.7ex, 1.5ex)$) -- ++(0, -3ex);
		  \draw[black!25, very thin] ($(n21)+(3.7ex,1.5ex)$) -- ++(0, -3ex);
		  \draw[black!25, very thin] ($(n22)+(3.7ex,1.5ex)$) -- ++(0, -3ex);

		\node[draw, green!100, thick, rectangle, minimum width = 3.4*8.5ex, minimum height=4ex] (d8) at (51.8em, -29em) {} ;
		  \node (n24) at (47em, -29em)    {\Large $k_{3,1}$} ;
		  \node (n25) at (50.2em, -29em)  {\Large $1$} ;
		  \node (n26) at (53.2em, -29em)  {\Large $p_{3,3}$} ;
		  \node (n27) at (56.2em, -29em)  {\Large $q_{3,1}$} ;
		  \draw[black!25, very thin] ($(n24)+(3.7ex, 1.5ex)$) -- ++(0, -3ex);
		  \draw[black!25, very thin] ($(n25)+(3.7ex,1.5ex)$) -- ++(0, -3ex);
		  \draw[black!25, very thin] ($(n26)+(3.7ex,1.5ex)$) -- ++(0, -3ex);

		\node[draw, green!100, thick, rectangle, minimum width = 3.4*8.5ex, minimum height=4ex] (d9) at (57.8em, -35em) {} ;
		  \node (n28) at (53em, -35em)    {\Large $k_{4,1}$} ;
		  \node (n29) at (56.2em, -35em)  {\Large $1$} ;
		  \node (n30) at (59.2em, -35em)  {\Large $p_{4,4}$} ;
		  \node (n31) at (62.2em, -35em)  {\Large $q_{4,1}$} ;
		  \draw[black!25, very thin] ($(n28)+(3.7ex, 1.5ex)$) -- ++(0, -3ex);
		  \draw[black!25, very thin] ($(n29)+(3.7ex,1.5ex)$) -- ++(0, -3ex);
		  \draw[black!25, very thin] ($(n30)+(3.7ex,1.5ex)$) -- ++(0, -3ex);

		\node[draw, green!100, thick, rectangle, minimum width = 3.4*6.5ex, minimum height=4ex] (d10) at (65.2em, -40em) {} ;
		  \node (n32) at (62em, -40em)    {\Large $k_{5,1}$} ;
		  \node (n33) at (65.2em, -40em)  {\Large $p_{5,5}$} ;
		  \node (n34) at (68.2em, -40em)  {\Large $q_{5,1}$} ;
		  \draw[black!25, very thin] ($(n32)+(3.7ex, 1.5ex)$) -- ++(0, -3ex);
		  \draw[black!25, very thin] ($(n33)+(3.7ex,1.5ex)$) -- ++(0, -3ex);

		\node[draw, green!100, fill=black!10, thick, rectangle, minimum width = 6.5ex, minimum height=25em] (d11) at (76em,-32em) {} ;
		  \node (n35) at (76em, -22em)    {\Large $x_{1}$} ;
		  \node (n36) at (76em, -27em)    {\Large $x_{2}$} ;
		  \node (n37) at (76em, -32em)    {\Large $x_{3}$} ;
		  \node (n38) at (76em, -37em)  {\Large $x_{4}$} ;
		  \node (n39) at (76em, -42em)  {\Large $x_{5}$} ;
		  \draw[black!25, very thin] ($(n35)-(3.0ex, 5.0ex)$) -- ++(6ex, 0);
		  \draw[black!25, very thin] ($(n36)-(3.0ex, 5.0ex)$) -- ++(6ex, 0);
		  \draw[black!25, very thin] ($(n37)-(3.0ex, 5.0ex)$) -- ++(6ex, 0);
		  \draw[black!25, very thin] ($(n38)-(3.0ex, 5.0ex)$) -- ++(6ex, 0);

		\node[draw, red!100, thick, rectangle, minimum width = 3.4*6.5ex, minimum height=4ex] (d12) at (5em, -40em) {} ;
		  \node (n40) at (5em, -40em)    {\Large $1$} ;


	%edges starting from d1
    \draw[hwbus, blue!100, ->]  (n1.south) -- ($(n1)-(0, 4.9ex)$) -- ($(n2)+(0,4.5ex)$) -- ($(n2)+(0,2ex)$) node[xshift=-0.5cm, yshift=0.4cm] {$/$} node[near start,above] {}  node[near end,right] {};

    \draw[hwbus, blue!100, ->]  (n1.south) -- ($(n1)-(0, 14.9ex)$) -- ($(n4)+(0,4ex)$) -- ($(n4.50)!(n4)!(n4.north)$) node[near start] {$/$} node[near start,above] {}  node[near end,right] {};

    \draw[hwbus, blue!100, ->]  (n1.south) -- ($(n1)-(0, 28.9ex)$) -- ($(n7)+(0,4ex)$) -- ($(n7.50)!(n7)!(n7.north)$) node[near start] {$/$} node[near start,above] {}  node[near end,right] {};

    \draw[hwbus, blue!100, ->]  (n1.south) -- ($(n1)-(0, 44.8ex)$) -- ($(n11)+(0,4ex)$) -- ($(n11.50)!(n11)!(n11.north)$) node[near start] {$/$} node[near start,above] {}  node[near end,right] {};
	
    \draw[hwbus, blue!100, ->]  (n1.south) -- ($(n1)-(0, 64.0ex)$) -- ($(n16)+(0,4ex)$) -- ($(n16.50)!(n16)!(n16.north)$) node[near start] {$/$} node[near start,above] {}  node[near end,right] {};

    \draw[hwbus, blue!100, ->]  (n1.south) -- ($(n1)-(0, 80.0ex)$) -- ($(n20)+(0,4ex)$) -- ($(n20.50)!(n20)!(n20.north)$) node[near start] {$/$} node[near start,above] {}  node[near end,right] {};

    \draw[hwbus, blue!100, ->]  (n1.south) -- ($(n1)-(0, 94.0ex)$) -- ($(n24)+(0,4ex)$) -- ($(n24.50)!(n24)!(n24.north)$) node[near start] {$/$} node[near start,above] {}  node[near end,right] {};
	
    \draw[hwbus, blue!100, ->]  (n1.south) -- ($(n1)-(0, 108.0ex)$) -- ($(n28)+(0,4ex)$) -- ($(n28.50)!(n28)!(n28.north)$) node[near start] {$/$} node[near start,above] {}  node[near end,right] {};

    \draw[hwbus, blue!100, ->]  (n1.south) -- ($(n1)-(0, 119.0ex)$) -- ($(n32)+(0,4ex)$) -- ($(n32.50)!(n32)!(n32.north)$) node[xshift=-10cm, yshift=0.2cm] {\Huge $/$} node[xshift=-10cm, yshift=-0.5cm] {\huge$(msb_{t_1},lsb_{t_1})$}  node[near end,right] {};

    \draw[hwbus, blue!100, ->]  (n1.south) -- ($(n1)-(0, 120.7ex)$) -- ($(n40)+(0,4ex)$) -- ($(n40.50)!(n40)!(n40.north)$) node[black!100, yshift=5cm,left] {$t_1(k+1)$} node[near start,above] {$/$}  node[near end,right] {};

	%edges starting from d2
    \draw[hwbus, blue1!100, ->]  (d2.south) -- ($(d2)-(0,1.2em)$) -- ($(n5)+(0,6.6ex)$) -- ($(n5.50)!(n5)!(n5.north)$) node[near start] {$/$} node[near start,above] {}  node[near end,right] {};

    \draw[hwbus, blue1!100, ->]  (d2.south) -- ($(d2)-(0,2em)$) -- ($(n8)-(31.7, -6.5ex)$) -- ($(n8)+(0,6.5ex)$) -- ($(n8.50)!(n8)!(n8.north)$) node[near start] {$/$} node[near start,above] {}  node[near end,right] {};

    \draw[hwbus, blue1!100, ->]  (d2.south) -- ($(d2)-(0,2em)$) -- ($(n12)-(50.0, -7ex)$) -- ($(n12)+(0,7ex)$) -- ($(n12.50)!(n12)!(n12.north)$) node[near start] {$/$} node[near start,above] {}  node[near end,right] {};

    \draw[hwbus, blue1!100, ->]  (d2.south) -- ($(d2)-(0,2em)$) -- ($(n17)-(70.5, -7ex)$) -- ($(n17)+(0,7ex)$) -- ($(n17.50)!(n17)!(n17.north)$) node[near start] {$/$} node[black!100, xshift=-12.6cm, yshift=2cm] {$t_2(k+1)$}  node[near end,right] {};

	%edges starting from d3
    \draw[hwbus, blue2!100, ->]  (d3.south) -- ($(d3)-(0,2.0em)$) -- ($(n9)+(0,9.5ex)$) -- ($(n9.50)!(n9)!(n9.north)$) node[near start] {$/$} node[near start,above] {}  node[near end,right] {};

    \draw[hwbus, blue2!100, ->]  (d3.south) -- ($(d3)-(0,9.0em)$) -- ($(n13)+(0,9.5ex)$) -- ($(n13.50)!(n13)!(n13.north)$) node[near start] {$/$} node[near start,above] {}  node[near end,right] {};

    \draw[hwbus, blue2!100, ->]  (d3.south) -- ($(d3)-(0,24.8em)$) -- ($(n21)+(0,7.5ex)$) -- ($(n21.50)!(n21)!(n21.north)$) node[near start] {$/$} node[black!100, xshift=-12.6cm, yshift=2cm] {$t_3(k+1)$}  node[near end,right] {};

	%edges starting from d4
    \draw[hwbus, blue3!100, ->]  (d4.south) -- ($(d4)-(0,1.5em)$) -- ($(n14)+(0,13.0ex)$) -- ($(n14.50)!(n14)!(n14.north)$) node[near start] {$/$} node[near start,above] {}  node[near end,right] {};

    \draw[hwbus, blue3!100, ->]  (d4.south) -- ($(d4)-(0,24.9em)$) -- ($(n25)+(0,7.5ex)$) -- ($(n25.50)!(n25)!(n25.north)$) node[near start] {$/$} node[black!100, xshift=-10.8cm, yshift=1.8cm] {$t_4(k+1)$}  node[near end,right] {};

	%edges starting from d5
    \draw[hwbus, blue4!100, ->]  (d5.south) -- ($(d5)-(0,24.0em)$) -- ($(n29)+(0,6.7ex)$) -- ($(n29.50)!(n29)!(n29.north)$) node[near start] {$/$} node[black!100, xshift=-9.4cm, yshift=1.8cm] {$t_5(k+1)$}  node[near end,right] {};

	%input of x register
    \draw[hwbus, green!100, ->]  (d6.south) -- ($(d6)-(0,2em)$) -- ($(n35)-(10,0ex)$) -- ($(n35)-(3,0)$) node[near start] {$/$} node[black!100, xshift=-0.9cm, yshift=0.5cm] {$x_1(k+1)$}  node[near end, right] {};
    \draw[hwbus, green!100, ->]  (d7.south) -- ($(d7)-(0,1.5em)$) -- ($(n36)-(9,0ex)$) -- ($(n36)-(3,0)$) node[near start] {$/$} node[black!100, xshift=-0.9cm, yshift=0.5cm] {$x_2(k+1)$}  node[near end, left] {};
    \draw[hwbus, green!100, ->]  (d8.south) -- ($(d8)-(0,1.5em)$) -- ($(n37)-(8,0ex)$) -- ($(n37)-(3,0)$) node[near start] {$/$} node[black!100, xshift=-0.9cm, yshift=0.5cm] {$x_3(k+1)$}  node[near end, left] {};
    \draw[hwbus, green!100, ->]  (d9.south) -- ($(d9)-(0,2em)$) -- ($(n38)-(7,0ex)$) -- ($(n38)-(3,0)$) node[near start] {$/$} node[black!100, xshift=-0.9cm, yshift=0.5cm] {$x_4(k+1)$}  node[near end, left] {};
    \draw[hwbus, green!100, ->]  (d10.south) -- ($(d10)-(0,2em)$) -- ($(n39)-(6,0ex)$) -- ($(n39)-(3,0)$) node[near start] {$/$} node[black!100, xshift=-0.9cm, yshift=0.5cm] {$x_5(k+1)$}  node[near end, left] {};

	%output of register to t
    \draw[hwbus, purple!100, ->]  (n35.east) -- ($(n35)+(10,0em)$) -- ($(n35)+(10,89ex)$) -- ($(n1)+(0,8ex)$) -- (n1.north) node[near start] {$/$} node[black!100, xshift=0.8cm, yshift=0.6cm] {$x_1(k)$}  node[near end,right] {};
    \draw[hwbus, purple!100, ->]  (n36.east) -- ($(n36)+(11,0em)$) -- ($(n36)+(11,98.5ex)$) -- ($(n3)+(0,15ex)$) -- (n3.north) node[near start] {$/$} node[black!100, xshift=0.8cm, yshift=1.7cm] {$x_2(k)$}  node[near end,right] {};
    \draw[hwbus, purple!100, ->]  (n37.east) -- ($(n37)+(12,0em)$) -- ($(n37)+(12,108ex)$) -- ($(n6)+(0,22ex)$) -- (n6.north) node[near start] {$/$} node[black!100, xshift=0.8cm, yshift=2.8cm] {$x_3(k)$}  node[near end,right] {};
    \draw[hwbus, purple!100, ->]  (n38.east) -- ($(n38)+(13,0em)$) -- ($(n38)+(13,117.5ex)$) -- ($(n10)+(0,33.5ex)$) -- (n10.north) node[near start] {$/$} node[black!100, xshift=0.8cm, yshift=4.5cm] {$x_4(k)$}  node[near end,right] {};
    \draw[hwbus, purple!100, ->]  (n39.east) -- ($(n39)+(14,0em)$) -- ($(n39)+(14,127ex)$) -- ($(n15)+(0,48ex)$) -- (n15.north) node[near start] {$/$} node[black!100, xshift=0.8cm, yshift=6.6cm] {$x_5(k)$}  node[near end,right] {};


	%output of register to x 
    \draw[hwbus, purple!100, ->]  (n35.east) -- ($(n35)+(10,0em)$) -- ($(n35)+(10,29ex)$) -- ($(n18)+(0,15ex)$) -- (n18.north) node[near start] {$/$} node[black!100, xshift=0.6cm, yshift=1.8cm] {$x_1(k)$}  node[near end,right] {};
    \draw[hwbus, purple!100, ->]  (n36.east) -- ($(n36)+(11,0em)$) -- ($(n36)+(11,39ex)$) -- ($(n22)+(0,29.5ex)$) -- (n22.north) node[near start] {$/$} node[black!100, xshift=0.6cm, yshift=4.0cm] {$x_2(k)$}  node[near end,right] {};
    \draw[hwbus, purple!100, ->]  (n37.east) -- ($(n37)+(12,0em)$) -- ($(n37)+(12,49ex)$) -- ($(n26)+(0,42.0ex)$) -- (n26.north) node[near start] {$/$} node[black!100, xshift=0.6cm, yshift=5.8cm] {$x_3(k)$}  node[near end,right] {};
    \draw[hwbus, purple!100, ->]  (n38.east) -- ($(n38)+(13,0em)$) -- ($(n38)+(13,59ex)$) -- ($(n30)+(0,54.5ex)$) -- (n30.north) node[near start] {$/$} node[black!100, xshift=0.6cm, yshift=7.7cm] {$x_4(k)$}  node[near end,right] {};
    \draw[hwbus, purple!100, ->]  (n39.east) -- ($(n39)+(14,0em)$) -- ($(n39)+(14,69ex)$) -- ($(n33)+(0,64.5ex)$) -- (n33.north) node[near start] {$/$} node[black!100, xshift=0.6cm, yshift=9.1cm] {$x_5(k)$}  node[near end,right] {};

	%inputs
    \draw[hwbus, orange1!100, ->]  ($(n19)+(0,35em)$) -- (n19.north) node[near start] {$/$} node[black!100, xshift=0.0cm, yshift=12.4cm] {$u_1(k)$}  node[near end,right] {};
    \draw[hwbus, orange1!100, ->]  ($(n23)+(0,42em)$) -- (n23.north) node[near start] {$/$} node[black!100, xshift=0.0cm, yshift=14.9cm] {$u_2(k)$}  node[near end,right] {};
    \draw[hwbus, orange1!100, ->]  ($(n27)+(0,48em)$) -- (n27.north) node[near start] {$/$} node[black!100, xshift=0.0cm, yshift=17.0cm] {$u_3(k)$}  node[near end,right] {};
    \draw[hwbus, orange1!100, ->]  ($(n31)+(0,54em)$) -- (n31.north) node[near start] {$/$} node[black!100, xshift=0.0cm, yshift=19.2cm] {$u_4(k)$}  node[near end,right] {};
    \draw[hwbus, orange1!100, ->]  ($(n34)+(0,59em)$) -- (n34.north) node[near start] {$/$} node[black!100, xshift=0.0cm, yshift=20.9cm] {$u_5(k)$}  node[near end,right] {};

	%outputs
    \draw[hwbus, red!100, ->]  (n40.south) -- ($(n40)-(0,3em)$) node[near start] {$/$} node[black!100, xshift=0.8cm, yshift=0.4cm] {$y_1(k)$}  node[near end,right] {};
 \end{tikzpicture}
}
%\end{center}
}
\end{center}
\caption{The FixRealKCM method when $x_i$ is split in 3 chunks   \label{fig:FixRealKCM}}
\end{figure}

	\begin{algorithm}[H]
	\For{i=1; i=Z.size(); i++} {
	 	row[ ]= Z[i][ ] //pick first row of Z \\
	 	\For {j=1; j=1; j=Z.size() j++} {
	 		assign(SOPC[i], row[j], TXU) //where TXU, is the indicator of the signal (this is just determined by the position of the coefficient)
	 	}
		Second pass for wiring.
	}
	\end{algorithm}
	

\subsection{Particular Forms}
	\subsection{ABCD Form}
	The ABCD Form can be considered as a degenerated form of the SIF, with nt=0.
	The algorithm will work in this case too.
	\subsection{$nx=0$}
	When $nx=0$, the interest of using implicit form is of course very limited.
	Still, the algorithm will work, allocating only SOPCs operators.
	
\subsection{Algorithm}

\subsection{Optimizations}
	\subsubsection{Sparse matrices}
		The Z matrix of a SIF might be sparse in some degenerated cases.
		So it is useful to remove zeros coefficients before allocating SOPCs.
		Indeed, it prevents useless inputs to be declared and can save a lot of hardware,
		although the HDL compiler might be able to optimize the hardware and remove "dead code".
		Anyway, it is healthy to keep a low compile time (either in flopoco or in the HDL compiler).
		Keeping the VHDL clean is more important, first for debugging issues, but also for comprehensiveness.
	\subsubsection{One entries}
		One entries in the Z matrix can be interpreted as simple wires instead of multiplications in the SOPC.
		So, we could eventually replace entries in SOPCs by simple additions with the result of the SOPC.
		Here, we should investigate to see what solution is the best in terms of hardware consumption
		(speed is not concerned here because the speed is determined by the length of the loop).




%\newpage
\section{Further work}
\label{Part3}
\subsection{Static analysis on coefficients}
	For now, we are performing detection on null coefficients.
	Detecting ones entries in the coefficients is an improvement very simple to implement.
	This will open the discussion of the efficiency of integrating or not these one entries in the SOPCs.
	Indeed, we don't know for now what choice is the best between:
	\begin{itemize}
		\item integrating the one entry in the SOPC
		\item keeping the one entry out of the SOPC and adding it to the result of the SOPC
	\end{itemize}

	Moreover, depending on the number of one entries, the implementations concerns might vary from one option to the other one in terms of logic consumption.


\subsection{Sub-filter detection}
	Detecting independent loops can be very interesting in the context of saving hardware.
	Indeed, considering a sub-filter with its own loop can permit to use its WCPG to compute the precisions for this sub-filter.
	We get then another problem with a precision specification that depends on the rest of the filter (logic after the sub-filter in the pipeline).

\subsection{Precision calculations improvement}
	In this work, we kept original calculations trying to adapt it to our context.
	However, lots of approximations are done through this calculations.
	Working on this part, going back over all these calculations could really improve the efficiency and the size of all implementations.


\subsection{File format re-specification}
	Specifying new formats could help to improve the visibility and the usability of this new operator.
	Working on standard specification techniques for this type of problems and adapting them in our context could greatly help the development of the SIF operator.
	New specifications could take some programming languages basic output functions, that is, allow the user not to waste hours formatting the matrices.
	Examples of compatibility with languages could look on:
	\begin{itemize}
		\item python interface/specification
		\item bash
		\item matlab, however the current syntax was derived from matlab outputs
	\end{itemize}

\subsection{Fourth Sub-part}



%\newpage
\section*{Conclusion}
	\addcontentsline{toc}{section}{Conclusion}
	This work tried to give an overview of LTI filters, and the considerations to take into account when trying to implement them.
	This report tried to adapt concepts and calculations from Lopez's and Hilaire's work to the context of generating architectures computing just right.
	This context is merging ideas from the communities of automatic, signal, and computer arithmetics, so a certain amount of time was spent on the unification of notations and conventions.
	Finally, the implementation of a parametric definition in the FloPoCo framework was started.
	This was permitted through the definition of a specification format and the integration of external code from LIP6.
	Further work will take into account many improvements and optimizations, such as:
	\begin{itemize}
		\item power of two removing
		\item sub-filter detection
		\item bounds and precision computation improvements
		\item format re-specification
	\end{itemize}

\bibliographystyle{plain}
\bibliography{report}  % main.bib is the name of the Bibliography in this case
\newpage
\begin{appendices}
\section{Notations and reminders about signal processing}
	%\addcontentsline{toc}{section}{Appendix: Notations and reminders about signal processing}
	LTI filters in general are usually defined as sums of products.
	Several quantities are useful to understand their characteristics

	\subsection{Notations}
		The entire report will use several notations and conventions:
		\begin{itemize}
			\item in signal processing, t is the common notation for continue time and k the notation for discrete time. This report keeps this convention.
			\item $\langle\langle\mathcal{H}\rangle\rangle _{WCPG}$ is the worst case peak gain of the filter $\mathcal{H}$
			\item y is an output variable
			\item x is a state variable
			\item t is an intermediate variable
			\item a bold symbol (for example $\boldsymbol{h}$) denotes a matrix, that may be a vector
			\item a normal symbol (for example $\varepsilon_{t_1}(k)$) denotes a single value
		\end{itemize}

	\subsection{Reminders about signal processing}
	\begin{thdef}\label{sig} (Signal)
		Generally, a signal is a temporal variable, which takes a value from $\mathbb{R}$ at each time t.
		x(t) denotes the value of the signal x at the instant t.
		When dealing with discrete time events, the time will be represented by k.
		Then x(k) is said to be a sample.
		$\{x(k)\}_{k \geq 0}$ denotes all the values possible for the signal x.
		The rest of this report, adresses vectors of signals $\textbf{x}$, where $\textbf{x}(k) \in \mathbb{R}^{n}$
	\end{thdef}

	\begin{thdef}\label{l_1} ($\ell^1$ norm)
		The $\ell^1$ norm of a scalar signal x, denoted $\|x\|_{\ell^1}$, is the sum of absolute values of $x(k)$ at each instant k:
		\begin{equation} \label{l1}
				\|x\|_{\ell^1}=\sum_{k=0}^{+\infty}|x(k)|
		\end{equation}
		This norm exists only if $x$ is $\ell^1$-sommable, that is, if and only if the equation \ref{l1} converges.
	\end{thdef}
	
	\begin{thdef}\label{l_1} ($\ell^2$ norm)
		The $\ell^2$ norm of a scalar signal x, denoted $\|x\|_{\ell^2}$, is defined as follows:
		\begin{equation} \label{l2}
				\|x\|_{\ell^2}=\sqrt{\sum_{k=0}^{+\infty}x(k)^2}
		\end{equation}
		This norm exists only if $x$ is square-sommable, that is, if and only if the equation \ref{l2} converges.
	\end{thdef}

	\begin{thdef}\label{l_inf} ($\ell^\infty$ norm)
		The $\ell^\infty$ norm of a scalar signal x, denoted $\|x\|_{\ell^\infty}$, is the smallest upper bound among all values (absolute values) possible for the signal x, that is:
		\begin{equation} \label{l3}
				\|x\|_{\ell^\infty}=\sup_{k\in \mathbb{N}}|x(k)|
		\end{equation}
	\end{thdef}

		\begin{thdef}\label{Dirac} (Dirac)
			The Dirac function, denoted $\delta$ describes an impulsion of infinite amplitude at a discrete time k (supposed to be instaneous):
			\begin{equation} \label{ir}
					\delta(k)=
					\begin{cases}
						1 & when\  k=0 \\
						0 & else
					\end{cases}
			\end{equation}
				
		\end{thdef}

\section{Table of contents}
\tableofcontents
\end{appendices}
%\TODO: talk about code: part title: "Implementation" (not in appendix)
\end{document}

