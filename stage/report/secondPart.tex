\subsection{}

In SOPCs architectures, the accuracy is deducible from the inputs/outputs specifications and the size of the constants.

This is described in %\paper{SOPCs}.

Dealing with feedback inputs, the question of precision is more compliated.
Indeed, when results loop back to inputs, as soon as we are in finite precision, the error is amplified by a certain amount, depending on the coefficients, at each pass through the filter.

The main idea to deimension such filters is to consider the total error as a single filter.
The result of this filter is then added to the perfect filter to get the final output.

In finite precision, sizes are constrained to be all the same. The demonstration of the size computation has been described in Lopez' PHD.
The idea now is to see what we can do in arbitrary precision.

Here we have to compute each size at each step of the computation. Indeed, the WCPG is not useful for the first part of a FIR, as it has no loop.
So we just need the WCPG for the second part, because it is just in this part of the circuit that there is a potential error amplification.

Direct and transposed forms are not directly transposable into SIF, but this problem is secondary.

\subsection{Algorithm}
\begin{figure}[!h]
\begin{center}
\scalebox{6}{  %\begin{center}
\resizebox{!}{80pt}{
  \begin{tikzpicture}

    \normalsize 
    \node[draw, blue!100,   ultra thick, rectangle, minimum width = 3.4*34.1ex, minimum height=4.0ex] (sel1)  at (42.0em, 45.7em) {} ;
    \node[draw, blue!100,   ultra thick, rectangle, minimum width = 3.4*34.1ex, minimum height=4.0ex] (sel2)  at (42.0em, 43.5em) {} ;
    \node[draw, blue!100,   ultra thick, rectangle, minimum width = 3.4*34.1ex, minimum height=4.0ex] (sel3)  at (42.0em, 41.4em) {} ;
    \node[draw, blue!100,   ultra thick, rectangle, minimum width = 3.4*34.1ex, minimum height=4.0ex] (sel4)  at (42.0em, 39.3em) {} ;
    \node[draw, blue!100,   ultra thick, rectangle, minimum width = 3.4*34.1ex, minimum height=4.0ex] (sel5)  at (42.0em, 37.1em) {} ;
    \node[draw, green!100,  ultra thick, rectangle, minimum width = 3.4*34.1ex, minimum height=4.0ex] (sel6)  at (42.0em, 35.2em) {} ;
    \node[draw, green!100,  ultra thick, rectangle, minimum width = 3.4*34.1ex, minimum height=4.0ex] (sel7)  at (42.0em, 32.9em) {} ;
    \node[draw, green!100,  ultra thick, rectangle, minimum width = 3.4*34.1ex, minimum height=4.0ex] (sel8)  at (42.0em, 30.8em) {} ;
    \node[draw, green!100,  ultra thick, rectangle, minimum width = 3.4*34.1ex, minimum height=4.0ex] (sel9)  at (42.0em, 28.6em) {} ;
    \node[draw, green!100,  ultra thick, rectangle, minimum width = 3.4*34.1ex, minimum height=4.0ex] (sel10) at (42.0em, 26.4em) {} ;
    \node[draw, red!100,    ultra thick, rectangle, minimum width = 3.4*34.1ex, minimum height=4.0ex] (sel11) at (42.0em, 24.5em) {} ;

	\normalsize
  \node at (98, 35em) {
	  \resizebox{!}{120pt}{
		$\begin{pmatrix}
			1       & 0       & 0       & 0       & 0       & m_{1,1} & 0       & 0       & 0       & 0       & 0       \\
			j_{2,1} & 1       & 0       & 0       & 0       & 0       & m_{2,2} & 0       & 0       & 0       & 0       \\
			j_{3,1} & j_{3,2} & 1       & 0       & 0       & 0       & 0       & m_{3,3} & 0       & 0       & 0       \\
			j_{4,1} & j_{4,2} & j_{4,3} & 1       & 0       & 0       & 0       & 0       & m_{4,4} & 0       & 0       \\
			j_{5,1} & j_{5,2} & j_{5,3} & j_{5,4} & 1       & 0       & 0       & 0       & 0       & m_{5,5} & 0       \\

			k_{1,1} & 1       & 0       & 0       & 0       & p_{1,1} & 0       & 0       & 0       & 0       & q_{1,1} \\
			k_{2,1} & 0       & 1       & 0       & 0       & 0       & p_{2,2} & 0       & 0       & 0       & q_{2,1} \\
			k_{3,1} & 0       & 0       & 1       & 0       & 0       & 0       & p_{3,3} & 0       & 0       & q_{3,1} \\
			k_{4,1} & 0       & 0       & 0       & 1       & 0       & 0       & 0       & p_{4,4} & 0       & q_{4,1} \\
			k_{5,1} & 0       & 0       & 0       & 0       & 0       & 0       & 0       & 0       & p_{5,5} & q_{5,1} \\

			1	    & 0       & 0       & 0       & 0       & 0       & 0       & 0       & 0       & 0       & 0       \\
		\end{pmatrix}$
	  }
  };

%	$\begin{pmatrix}
%		1       & 0       & 0       & 0       & 0       & m_{1,1} & m_{1,2} & m_{1,3} & m_{1,4} & m_{1,5} & n_{1,1} \\
%		j_{2,1} & 1       & 0       & 0       & 0       & m_{2,1} & m_{2,2} & m_{2,3} & m_{2,4} & m_{2,5} & n_{2,1} \\
%		j_{3,1} & j_{3,2} & 1       & 0       & 0       & m_{3,1} & m_{3,2} & m_{3,3} & m_{3,4} & m_{3,5} & n_{3,1} \\
%		j_{4,1} & j_{4,2} & j_{4,3} & 1       & 0       & m_{4,1} & m_{4,2} & m_{4,3} & m_{4,4} & m_{4,5} & n_{4,1} \\
%		j_{5,1} & j_{5,2} & j_{5,3} & j_{5,4} & 1       & m_{5,1} & m_{5,2} & m_{5,3} & m_{5,4} & m_{5,5} & n_{5,1} \\
%
%		k_{1,1} & k_{1,1} & k_{1,1} & k_{1,1} & k_{1,1} & p_{1,1} & p_{1,1} & p_{1,1} & p_{1,1} & p_{1,1} & q_{1,1} \\
%		k_{2,1} & k_{2,1} & k_{2,1} & k_{2,1} & k_{2,1} & p_{1,1} & p_{1,1} & p_{1,1} & p_{1,1} & p_{1,1} & q_{2,1} \\
%		k_{3,1} & k_{3,1} & k_{3,1} & k_{3,1} & k_{3,1} & p_{1,1} & p_{1,1} & p_{1,1} & p_{1,1} & p_{1,1} & q_{3,1} \\
%		k_{4,1} & k_{4,1} & k_{4,1} & k_{4,1} & k_{4,1} & p_{1,1} & p_{1,1} & p_{1,1} & p_{1,1} & p_{1,1} & q_{4,1} \\
%		k_{5,1} & k_{5,1} & k_{5,1} & k_{5,1} & k_{5,1} & p_{1,1} & p_{1,1} & p_{1,1} & p_{1,1} & p_{1,1} & q_{5,1} \\
%                                                       
%		l_{5,1} & l_{5,1} & l_{5,1} & l_{5,1} & l_{5,1} & r_{1,1} & r_{1,2} & r_{1,3} & r_{1,4} & r_{1,5} & s_{1,1} \\
%	\end{pmatrix}$

		\node[draw, blue!100, thick, rectangle, minimum width = 3.4*5ex, minimum height=4ex] (d1) at (5em, 13em) {} ;
		  \node (n1) at (5em, 13em)  {\Large $m_{1,1}$} ;
		  \draw[black!25, very thin] ($(n1)-(3.7ex,-1.5ex)$) -- ++(0, -3ex);
		  \draw[black!25, very thin] ($(n1)+(3.7ex, 1.5ex)$) -- ++(0, -3ex);

		\node[draw, blue!100, thick, rectangle, minimum width = 3.4*4.4ex, minimum height=4ex] (d2) at (9.6em,9em) {} ;
		  \node (n2) at (8em, 9em)  {\large $-j_{2,1}$} ;
		  \node (n3) at (11.2em, 9em)  {\Large $m_{2,2}$} ;
		  \draw[black!25, very thin] ($(n2)+(3.7ex, 1.5ex)$) -- ++(0, -3ex);

		\node[draw, blue!100, thick, rectangle, minimum width = 3.4*6.5ex, minimum height=4ex] (d3) at (15.2em,5em) {} ;
		  \node (n4) at (12em, 5em)    {\large $-j_{3,1}$} ;
		  \node (n5) at (15.2em, 5em)  {\large $-j_{3,2}$} ;
		  \node (n6) at (18.2em, 5em)  {\Large $m_{3,3}$} ;
		  \draw[black!25, very thin] ($(n4)+(3.7ex, 1.5ex)$) -- ++(0, -3ex);
		  \draw[black!25, very thin] ($(n5)+(3.7ex,1.5ex)$) -- ++(0, -3ex);

		\node[draw, blue!100, thick, rectangle, minimum width = 3.4*8.5ex, minimum height=4ex] (d4) at (24.6em,-1em) {} ;
		  \node (n7) at (20em, -1em)     {\large $-j_{4,1}$} ;
		  \node (n8) at (23.2em, -1em)   {\large $-j_{4,2}$} ;
		  \node (n9) at (26.2em, -1em)   {\large $-j_{4,3}$} ;
		  \node (n10) at (29.2em, -1em)  {\Large $m_{4,4}$} ;
		  \draw[black!25, very thin] ($(n7)+(3.7ex, 1.5ex)$) -- ++(0, -3ex);
		  \draw[black!25, very thin] ($(n8)+(3.7ex,1.5ex)$) -- ++(0, -3ex);
		  \draw[black!25, very thin] ($(n9)+(3.7ex,1.5ex)$) -- ++(0, -3ex);

		\node[draw, blue!100, thick, rectangle, minimum width = 3.4*10.6ex, minimum height=4ex] (d5) at (34.2em, -8em) {} ;
		  \node (n11) at (28em, -8em)    {\large $-j_{5,1}$} ;
		  \node (n12) at (31.2em, -8em)  {\large $-j_{5,2}$} ;
		  \node (n13) at (34.2em, -8em)  {\large $-j_{5,3}$} ;
		  \node (n14) at (37.2em, -8em)  {\large $-j_{5,4}$} ;
		  \node (n15) at (40.2em, -8em)  {\Large $m_{5,5}$} ;
		  \draw[black!25, very thin] ($(n11)+(3.7ex, 1.5ex)$) -- ++(0, -3ex);
		  \draw[black!25, very thin] ($(n12)+(3.7ex,1.5ex)$) -- ++(0, -3ex);
		  \draw[black!25, very thin] ($(n13)+(3.7ex,1.5ex)$) -- ++(0, -3ex);
		  \draw[black!25, very thin] ($(n14)+(3.7ex,1.5ex)$) -- ++(0, -3ex);

		\node[draw, green!100, thick, rectangle, minimum width = 3.4*8.5ex, minimum height=4ex] (d6) at (41.7em, -16em) {} ;
		  \node (n16) at (37em, -16em)    {\Large $k_{1,1}$} ;
		  \node (n17) at (40.2em, -16em)  {\Large $1$} ;
		  \node (n18) at (43.2em, -16em)  {\Large $p_{1,1}$} ;
		  \node (n19) at (46.2em, -16em)  {\Large $q_{1,1}$} ;
		  \draw[black!25, very thin] ($(n16)+(3.7ex, 1.5ex)$) -- ++(0, -3ex);
		  \draw[black!25, very thin] ($(n17)+(3.7ex,1.5ex)$) -- ++(0, -3ex);
		  \draw[black!25, very thin] ($(n18)+(3.7ex,1.5ex)$) -- ++(0, -3ex);

		\node[draw, green!100, thick, rectangle, minimum width = 3.4*8.5ex, minimum height=4ex] (d7) at (46.7em, -23em) {} ;
		  \node (n20) at (42em, -23em)    {\Large $k_{2,1}$} ;
		  \node (n21) at (45.2em, -23em)  {\Large $1$} ;
		  \node (n22) at (48.2em, -23em)  {\Large $p_{2,2}$} ;
		  \node (n23) at (51.2em, -23em)  {\Large $q_{2,1}$} ;
		  \draw[black!25, very thin] ($(n20)+(3.7ex, 1.5ex)$) -- ++(0, -3ex);
		  \draw[black!25, very thin] ($(n21)+(3.7ex,1.5ex)$) -- ++(0, -3ex);
		  \draw[black!25, very thin] ($(n22)+(3.7ex,1.5ex)$) -- ++(0, -3ex);

		\node[draw, green!100, thick, rectangle, minimum width = 3.4*8.5ex, minimum height=4ex] (d8) at (51.8em, -29em) {} ;
		  \node (n24) at (47em, -29em)    {\Large $k_{3,1}$} ;
		  \node (n25) at (50.2em, -29em)  {\Large $1$} ;
		  \node (n26) at (53.2em, -29em)  {\Large $p_{3,3}$} ;
		  \node (n27) at (56.2em, -29em)  {\Large $q_{3,1}$} ;
		  \draw[black!25, very thin] ($(n24)+(3.7ex, 1.5ex)$) -- ++(0, -3ex);
		  \draw[black!25, very thin] ($(n25)+(3.7ex,1.5ex)$) -- ++(0, -3ex);
		  \draw[black!25, very thin] ($(n26)+(3.7ex,1.5ex)$) -- ++(0, -3ex);

		\node[draw, green!100, thick, rectangle, minimum width = 3.4*8.5ex, minimum height=4ex] (d9) at (57.8em, -35em) {} ;
		  \node (n28) at (53em, -35em)    {\Large $k_{4,1}$} ;
		  \node (n29) at (56.2em, -35em)  {\Large $1$} ;
		  \node (n30) at (59.2em, -35em)  {\Large $p_{4,4}$} ;
		  \node (n31) at (62.2em, -35em)  {\Large $q_{4,1}$} ;
		  \draw[black!25, very thin] ($(n28)+(3.7ex, 1.5ex)$) -- ++(0, -3ex);
		  \draw[black!25, very thin] ($(n29)+(3.7ex,1.5ex)$) -- ++(0, -3ex);
		  \draw[black!25, very thin] ($(n30)+(3.7ex,1.5ex)$) -- ++(0, -3ex);

		\node[draw, green!100, thick, rectangle, minimum width = 3.4*6.5ex, minimum height=4ex] (d10) at (65.2em, -40em) {} ;
		  \node (n32) at (62em, -40em)    {\Large $k_{5,1}$} ;
		  \node (n33) at (65.2em, -40em)  {\Large $p_{5,5}$} ;
		  \node (n34) at (68.2em, -40em)  {\Large $q_{5,1}$} ;
		  \draw[black!25, very thin] ($(n32)+(3.7ex, 1.5ex)$) -- ++(0, -3ex);
		  \draw[black!25, very thin] ($(n33)+(3.7ex,1.5ex)$) -- ++(0, -3ex);

		\node[draw, green!100, fill=black!10, thick, rectangle, minimum width = 6.5ex, minimum height=25em] (d11) at (76em,-32em) {} ;
		  \node (n35) at (76em, -22em)    {\Large $x_{1}$} ;
		  \node (n36) at (76em, -27em)    {\Large $x_{2}$} ;
		  \node (n37) at (76em, -32em)    {\Large $x_{3}$} ;
		  \node (n38) at (76em, -37em)  {\Large $x_{4}$} ;
		  \node (n39) at (76em, -42em)  {\Large $x_{5}$} ;
		  \draw[black!25, very thin] ($(n35)-(3.0ex, 5.0ex)$) -- ++(6ex, 0);
		  \draw[black!25, very thin] ($(n36)-(3.0ex, 5.0ex)$) -- ++(6ex, 0);
		  \draw[black!25, very thin] ($(n37)-(3.0ex, 5.0ex)$) -- ++(6ex, 0);
		  \draw[black!25, very thin] ($(n38)-(3.0ex, 5.0ex)$) -- ++(6ex, 0);

		\node[draw, red!100, thick, rectangle, minimum width = 3.4*6.5ex, minimum height=4ex] (d12) at (5em, -40em) {} ;
		  \node (n40) at (5em, -40em)    {\Large $1$} ;


	%edges starting from d1
    \draw[hwbus, blue!100, ->]  (n1.south) -- ($(n1)-(0, 4.9ex)$) -- ($(n2)+(0,4.5ex)$) -- ($(n2)+(0,2ex)$) node[xshift=-0.5cm, yshift=0.4cm] {$/$} node[near start,above] {}  node[near end,right] {};

    \draw[hwbus, blue!100, ->]  (n1.south) -- ($(n1)-(0, 14.9ex)$) -- ($(n4)+(0,4ex)$) -- ($(n4.50)!(n4)!(n4.north)$) node[near start] {$/$} node[near start,above] {}  node[near end,right] {};

    \draw[hwbus, blue!100, ->]  (n1.south) -- ($(n1)-(0, 28.9ex)$) -- ($(n7)+(0,4ex)$) -- ($(n7.50)!(n7)!(n7.north)$) node[near start] {$/$} node[near start,above] {}  node[near end,right] {};

    \draw[hwbus, blue!100, ->]  (n1.south) -- ($(n1)-(0, 44.8ex)$) -- ($(n11)+(0,4ex)$) -- ($(n11.50)!(n11)!(n11.north)$) node[near start] {$/$} node[near start,above] {}  node[near end,right] {};
	
    \draw[hwbus, blue!100, ->]  (n1.south) -- ($(n1)-(0, 64.0ex)$) -- ($(n16)+(0,4ex)$) -- ($(n16.50)!(n16)!(n16.north)$) node[near start] {$/$} node[near start,above] {}  node[near end,right] {};

    \draw[hwbus, blue!100, ->]  (n1.south) -- ($(n1)-(0, 80.0ex)$) -- ($(n20)+(0,4ex)$) -- ($(n20.50)!(n20)!(n20.north)$) node[near start] {$/$} node[near start,above] {}  node[near end,right] {};

    \draw[hwbus, blue!100, ->]  (n1.south) -- ($(n1)-(0, 94.0ex)$) -- ($(n24)+(0,4ex)$) -- ($(n24.50)!(n24)!(n24.north)$) node[near start] {$/$} node[near start,above] {}  node[near end,right] {};
	
    \draw[hwbus, blue!100, ->]  (n1.south) -- ($(n1)-(0, 108.0ex)$) -- ($(n28)+(0,4ex)$) -- ($(n28.50)!(n28)!(n28.north)$) node[near start] {$/$} node[near start,above] {}  node[near end,right] {};

    \draw[hwbus, blue!100, ->]  (n1.south) -- ($(n1)-(0, 119.0ex)$) -- ($(n32)+(0,4ex)$) -- ($(n32.50)!(n32)!(n32.north)$) node[near start] {$/$} node[near start,above] {}  node[near end,right] {};

    \draw[hwbus, blue!100, ->]  (n1.south) -- ($(n1)-(0, 120.7ex)$) -- ($(n40)+(0,4ex)$) -- ($(n40.50)!(n40)!(n40.north)$) node[yshift=5cm,left] {$t_1(k+1)$} node[near start,above] {$/$}  node[near end,right] {};

	%edges starting from d2
    \draw[hwbus, blue1!100, ->]  (d2.south) -- ($(d2)-(0,1.2em)$) -- ($(n5)+(0,6.6ex)$) -- ($(n5.50)!(n5)!(n5.north)$) node[near start] {$/$} node[near start,above] {}  node[near end,right] {};

    \draw[hwbus, blue1!100, ->]  (d2.south) -- ($(d2)-(0,2em)$) -- ($(n8)-(31.7, -6.5ex)$) -- ($(n8)+(0,6.5ex)$) -- ($(n8.50)!(n8)!(n8.north)$) node[near start] {$/$} node[near start,above] {}  node[near end,right] {};

    \draw[hwbus, blue1!100, ->]  (d2.south) -- ($(d2)-(0,2em)$) -- ($(n12)-(50.0, -7ex)$) -- ($(n12)+(0,7ex)$) -- ($(n12.50)!(n12)!(n12.north)$) node[near start] {$/$} node[near start,above] {}  node[near end,right] {};

    \draw[hwbus, blue1!100, ->]  (d2.south) -- ($(d2)-(0,2em)$) -- ($(n17)-(70.5, -7ex)$) -- ($(n17)+(0,7ex)$) -- ($(n17.50)!(n17)!(n17.north)$) node[near start] {$/$} node[xshift=-12.6cm, yshift=2cm] {$t_2(k+1)$}  node[near end,right] {};

	%edges starting from d3
    \draw[hwbus, blue2!100, ->]  (d3.south) -- ($(d3)-(0,2.0em)$) -- ($(n9)+(0,9.5ex)$) -- ($(n9.50)!(n9)!(n9.north)$) node[near start] {$/$} node[near start,above] {}  node[near end,right] {};

    \draw[hwbus, blue2!100, ->]  (d3.south) -- ($(d3)-(0,9.0em)$) -- ($(n13)+(0,9.5ex)$) -- ($(n13.50)!(n13)!(n13.north)$) node[near start] {$/$} node[near start,above] {}  node[near end,right] {};

    \draw[hwbus, blue2!100, ->]  (d3.south) -- ($(d3)-(0,24.8em)$) -- ($(n21)+(0,7.5ex)$) -- ($(n21.50)!(n21)!(n21.north)$) node[near start] {$/$} node[xshift=-12.6cm, yshift=2cm] {$t_1(k+1)$}  node[near end,right] {};

	%edges starting from d4
    \draw[hwbus, blue3!100, ->]  (d4.south) -- ($(d4)-(0,1.5em)$) -- ($(n14)+(0,13.0ex)$) -- ($(n14.50)!(n14)!(n14.north)$) node[near start] {$/$} node[near start,above] {}  node[near end,right] {};

    \draw[hwbus, blue3!100, ->]  (d4.south) -- ($(d4)-(0,24.9em)$) -- ($(n25)+(0,7.5ex)$) -- ($(n25.50)!(n25)!(n25.north)$) node[near start] {$/$} node[xshift=-10.8cm, yshift=1.8cm] {$t_1(k+1)$}  node[near end,right] {};

	%edges starting from d5
    \draw[hwbus, blue4!100, ->]  (d5.south) -- ($(d5)-(0,24.0em)$) -- ($(n29)+(0,6.7ex)$) -- ($(n29.50)!(n29)!(n29.north)$) node[near start] {$/$} node[xshift=-9.4cm, yshift=1.8cm] {$t_1(k+1)$}  node[near end,right] {};

	%input of x register
    \draw[hwbus, green!100, ->]  (d6.south) -- ($(d6)-(0,2em)$) -- ($(n35)-(10,0ex)$) -- ($(n35)-(3,0)$) node[near start] {$/$} node[xshift=-0.9cm, yshift=0.5cm] {$x_1(k+1)$}  node[near end, right] {};
    \draw[hwbus, green!100, ->]  (d7.south) -- ($(d7)-(0,1.5em)$) -- ($(n36)-(9,0ex)$) -- ($(n36)-(3,0)$) node[near start] {$/$} node[xshift=-0.9cm, yshift=0.5cm] {$x_2(k+1)$}  node[near end, left] {};
    \draw[hwbus, green!100, ->]  (d8.south) -- ($(d8)-(0,1.5em)$) -- ($(n37)-(8,0ex)$) -- ($(n37)-(3,0)$) node[near start] {$/$} node[xshift=-0.9cm, yshift=0.5cm] {$x_3(k+1)$}  node[near end, left] {};
    \draw[hwbus, green!100, ->]  (d9.south) -- ($(d9)-(0,2em)$) -- ($(n38)-(7,0ex)$) -- ($(n38)-(3,0)$) node[near start] {$/$} node[xshift=-0.9cm, yshift=0.5cm] {$x_4(k+1)$}  node[near end, left] {};
    \draw[hwbus, green!100, ->]  (d10.south) -- ($(d10)-(0,2em)$) -- ($(n39)-(6,0ex)$) -- ($(n39)-(3,0)$) node[near start] {$/$} node[xshift=-0.9cm, yshift=0.5cm] {$x_5(k+1)$}  node[near end, left] {};

	%output of register to t
    \draw[hwbus, purple!100, ->]  (n35.east) -- ($(n35)+(10,0em)$) -- ($(n35)+(10,89ex)$) -- ($(n1)+(0,8ex)$) -- (n1.north) node[near start] {$/$} node[xshift=0.8cm, yshift=0.6cm] {$x_1(k)$}  node[near end,right] {};
    \draw[hwbus, purple!100, ->]  (n36.east) -- ($(n36)+(11,0em)$) -- ($(n36)+(11,98.5ex)$) -- ($(n3)+(0,15ex)$) -- (n3.north) node[near start] {$/$} node[xshift=0.8cm, yshift=1.7cm] {$x_2(k)$}  node[near end,right] {};
    \draw[hwbus, purple!100, ->]  (n37.east) -- ($(n37)+(12,0em)$) -- ($(n37)+(12,108ex)$) -- ($(n6)+(0,22ex)$) -- (n6.north) node[near start] {$/$} node[xshift=0.8cm, yshift=2.8cm] {$x_3(k)$}  node[near end,right] {};
    \draw[hwbus, purple!100, ->]  (n38.east) -- ($(n38)+(13,0em)$) -- ($(n38)+(13,117.5ex)$) -- ($(n10)+(0,33.5ex)$) -- (n10.north) node[near start] {$/$} node[xshift=0.8cm, yshift=4.5cm] {$x_4(k)$}  node[near end,right] {};
    \draw[hwbus, purple!100, ->]  (n39.east) -- ($(n39)+(14,0em)$) -- ($(n39)+(14,127ex)$) -- ($(n15)+(0,48ex)$) -- (n15.north) node[near start] {$/$} node[xshift=0.8cm, yshift=6.6cm] {$x_5(k)$}  node[near end,right] {};


	%output of register to x 
    \draw[hwbus, purple!100, ->]  (n35.east) -- ($(n35)+(10,0em)$) -- ($(n35)+(10,29ex)$) -- ($(n18)+(0,15ex)$) -- (n18.north) node[near start] {$/$} node[xshift=0.6cm, yshift=1.8cm] {$x_1(k)$}  node[near end,right] {};
    \draw[hwbus, purple!100, ->]  (n36.east) -- ($(n36)+(11,0em)$) -- ($(n36)+(11,39ex)$) -- ($(n22)+(0,29.5ex)$) -- (n22.north) node[near start] {$/$} node[xshift=0.6cm, yshift=4.0cm] {$x_2(k)$}  node[near end,right] {};
    \draw[hwbus, purple!100, ->]  (n37.east) -- ($(n37)+(12,0em)$) -- ($(n37)+(12,49ex)$) -- ($(n26)+(0,42.0ex)$) -- (n26.north) node[near start] {$/$} node[xshift=0.6cm, yshift=5.8cm] {$x_3(k)$}  node[near end,right] {};
    \draw[hwbus, purple!100, ->]  (n38.east) -- ($(n38)+(13,0em)$) -- ($(n38)+(13,59ex)$) -- ($(n30)+(0,54.5ex)$) -- (n30.north) node[near start] {$/$} node[xshift=0.6cm, yshift=7.7cm] {$x_4(k)$}  node[near end,right] {};
    \draw[hwbus, purple!100, ->]  (n39.east) -- ($(n39)+(14,0em)$) -- ($(n39)+(14,69ex)$) -- ($(n33)+(0,64.5ex)$) -- (n33.north) node[near start] {$/$} node[xshift=0.6cm, yshift=9.1cm] {$x_5(k)$}  node[near end,right] {};

	%inputs
    \draw[hwbus, orange1!100, ->]  ($(n19)+(0,35em)$) -- (n19.north) node[near start] {$/$} node[xshift=0.0cm, yshift=12.4cm] {$u_1(k)$}  node[near end,right] {};
    \draw[hwbus, orange1!100, ->]  ($(n23)+(0,42em)$) -- (n23.north) node[near start] {$/$} node[xshift=0.0cm, yshift=14.9cm] {$u_2(k)$}  node[near end,right] {};
    \draw[hwbus, orange1!100, ->]  ($(n27)+(0,48em)$) -- (n27.north) node[near start] {$/$} node[xshift=0.0cm, yshift=17.0cm] {$u_3(k)$}  node[near end,right] {};
    \draw[hwbus, orange1!100, ->]  ($(n31)+(0,54em)$) -- (n31.north) node[near start] {$/$} node[xshift=0.0cm, yshift=19.2cm] {$u_4(k)$}  node[near end,right] {};
    \draw[hwbus, orange1!100, ->]  ($(n34)+(0,59em)$) -- (n34.north) node[near start] {$/$} node[xshift=0.0cm, yshift=20.9cm] {$u_5(k)$}  node[near end,right] {};

	%outputs
    \draw[hwbus, red!100, ->]  (n40.south) -- ($(n40)-(0,3em)$) node[near start] {$/$} node[xshift=0.8cm, yshift=0.4cm] {$x_5(k)$}  node[near end,right] {};
 \end{tikzpicture}
}
%\end{center}
}
\end{center}
\caption{The FixRealKCM method when $x_i$ is split in 3 chunks   \label{fig:FixRealKCM}}
\end{figure}

	\begin{algorithm}[H]
	\For{i=1; i=Z.size(); i++} {
	 	row[ ]= Z[i][ ] //pick first row of Z \\
	 	\For {j=1; j=1; j=Z.size() j++} {
	 		assign(SOPC[i], row[j], TXU) //where TXU, is the indicator of the signal (this is just determined by the position of the coefficient)
	 	}
		Second pass for wiring.
	}
	\end{algorithm}
	

\subsection{Particular Forms}
	\subsection{ABCD Form}
	The ABCD Form can be considered as a degenerated form of the SIF, with nt=0.
	The algorithm will work in this case too.
	\subsection{$nx=0$}
	When $nx=0$, the interest of using implicit form is of course very limited.
	Still, the algorithm will work, allocating only SOPCs operators.
	
\subsection{Algorithm}

\subsection{Optimizations}
	\subsubsection{Sparse matrices}
		The Z matrix of a SIF might be sparse in some degenerated cases.
		So it is useful to remove zeros coefficients before allocating SOPCs.
		Indeed, it prevents useless inputs to be declared and can save a lot of hardware,
		although the HDL compiler might be able to optimize the hardware and remove "dead code".
		Anyway, it is healthy to keep a low compile time (either in flopoco or in the HDL compiler).
		Keeping the VHDL clean is more important, first for debugging issues, but also for comprehensiveness.
	\subsubsection{One entries}
		One entries in the Z matrix can be interpreted as simple wires instead of multiplications in the SOPC.
		So, we could eventually replace entries in SOPCs by simple additions with the result of the SOPC.
		Here, we should investigate to see what solution is the best in terms of hardware consumption
		(speed is not concerned here because the speed is determined by the length of the loop).


