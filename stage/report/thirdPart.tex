\subsection{Static analysis on coefficients}
	For now, detection on null coefficients is already achieved.
	Detecting power of two entries in the coefficients seem to be a quite simple improvement to implement.
	This will open the discussion of the efficiency of integrating or not these one entries in the SOPCs, because the best choice is not known between:
	\begin{itemize}
		\item integrating the power of two entry in the SOPC
		\item keeping the power of two entry out of the SOPC and adding it to the result of the SOPC
	\end{itemize}

	Moreover, depending on the number of power of two entries, the implementations concerns might vary from one option to the other one in terms of logic consumption.


\subsection{Sub-filter detection}
	Detecting independent loops can be very interesting in the context of saving hardware.
	Considering a sub-filter with its own loop can permit to use its WCPG to compute the precisions for this sub-filter.
	Thus leads ot another resolution with a precision specification that depends on the rest of the filter (logic after the sub-filter in the pipeline).

\subsection{Precision calculations improvement}
	This work kept original calculations trying to adapt them to our context.
	However, lots of approximations are done through those calculations.
	Working on this part, going back over all the calculations could really improve the efficiency and size of all implementations.


\subsection{File format re-specification}
	Specifying new formats could help to improve the visibility and the usability of this new operator.
	The idea is to avoid specification output formatting operations for the user.
	So sticking to the basic output of Python and Matlab matrices would be much comfortable for the user.

