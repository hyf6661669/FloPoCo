\subsection{Static analysis on coefficients}
	Detection of and removal of null coefficients is already implemented.
	Detecting power of two entries in the coefficients seem to be a quite simple improvement to implement.
	However, this part should be left to the implementation of SOPCs, and so KCM multipliers.
	Indeed, pushing this detection at the lowest level is easy and gives the opportunity to the lower operators to give a feedback about their actual accuracy.
	Detecting a power of two in a KCM multiplier with feedback on precision will help to build better calculations on msbs and lsbs.
	Moreover, doing the job in the KCM operator will prevent spending compile time analysing the list of coefficients.



\subsection{Sub-filter detection}
	Detecting independent loops can be very interesting in the context of saving hardware.
	Considering a sub-filter with its own loop can permit to use its WCPG to compute the precisions for this sub-filter.
	This leads to another resolution with a precision specification that depends on the rest of the filter (logic after the sub-filter in the pipeline).
	However, this optimization can be done at the front-end step, which leads to an open question.
	In fact, detecting such loops has issues both in front-end and in backend, and is a non-trivial problem.
	So it remains as an open question for future investigations.

\subsection{Precision calculations improvement}
	This work kept original calculations from Lopez's PhD \cite{lopez}, trying to adapt them to our context.
	However, lots of approximations are done through those calculations.
	Working on this part, going back over all the calculations could really improve the efficiency and size of all hardware implementations.


\subsection{File format re-specification}
	Specifying new formats could help to improve the visibility and the usability of this new operator.
	The idea is to avoid specification output formatting operations for the user.
	In the future, the present implementation should recognize input matrices formated in the Python or Matlab formats.
